%!TEX root = ../lectures.tex

\lecture[October 8th, 2019]{The Space of Meromorphic Functions}

\topic{The topology of $C(G, \C_\infty)$}

Recall how we model the Riemann sphere\index{Riemann sphere} $\C_\infty = \C \cup \Set{\infty}$ as a sphere centred on the origin of the origin of the complex plane, and we identify the north pole of the sphere as $\infty$, and if we imagine a line connecting this north pole with any given point $z$ on the complex plane, where this line intersects the sphere, say $Z$, is the corresponding point.
We have illustrated this in \autoref{l15:figa}.

\begin{figure}
	\centering
	\includegraphics[width=\textwidth]{figures/l15figa.tikz}

	\caption{\label{l15:figa} A model of the Riemann sphere, identifying points $z_1, z_2 \in \C$ with points $Z_1, Z_2 \in \C_\infty$.}
\end{figure}

If we wish to study continuous functions from $G$ to $\C_\infty$, we first need a metric on $\C_\infty$.
We define this metric $d$ by
\[
	d(z_1, z_2) = d(Z_1, Z_2)
\]
for $z_1, z_2 \in \C$, where $Z_1$ and $Z_2$ are the points corresponding to $z_1$ and $z_2$ on the sphere, and the right-hand side distance is measured in the usual Euclidean way in $\R^3$.

Then in all, we define the metric on $\C_\infty$ by
\begin{items}
	\item\label{ddefi} if $z_1, z_2 \in \C$, then $\displaystyle d(z_1, z_2) = \frac{2 \abs{z_1 - z_2}}{(1 + \abs{z_1}^2)^{1/2} (1 + \abs{z_2}^2)^{1/2}}$ (this is just $d(Z_1, Z_2)$ written out); and
	\item if $z \in \C$, then $\displaystyle d(z, \infty) = \frac{2}{(1 + \abs{z}^2)^{1/2}}$.
\end{items}

This metric has an interesting property:
for $z_1, z_2 \neq 0$,
\[
	d(z_1, z_2) = d\left ( \frac{1}{z_1}, \frac{1}{z_2} \right ),
\]
and similarly for $z \neq 0$,
\[
	d(z, 0) = d \left ( \frac{1}{z}, \infty \right ).
\]
This is a consequence of $z$ and $1/z$ corresponding to symmetric points in the upper and lower hemispheres in this model.

\begin{remark}
	In order to distinguish the topology on the complex plane and the topology on the Riemann sphere, we will (as before) use $B(a, r)$ to mean an open ball in $(\C, \abs{})$, and $B_\infty(a, r)$ to mean a ball in the Riemann sphere $(\C_\infty, d)$.
\end{remark}

The good news is that these topologies are essentially the same (barring the tricky point at infinity):

\begin{proposition}\label{prop7.12}
	\begin{items}
		\item\label{p7.12i} Given $a \in \C$ and $r > 0$, there exists $\rho > 0$ such that $B_\infty(a, \rho) \subset B(a, r)$ (so open in $\abs{}$ implies open in $d$);
		\item\label{p7.12ii} Given $a \in \C$ and $\rho > 0$, there exists $r > 0$ such that $B(a, r) \subset B_\infty(a, \rho)$ (so open in $d$ implies open in $\abs{}$);
		\item\label{p7.12iii} Given $\rho > 0$, there exists a compact set $K \subset \C$ such that $\C_\infty \setminus K \subset B_\infty(\infty, \rho)$; and
		\item\label{p7.12iv} Given a compact set $K \subset \C$, there exists $\rho > 0$ such that $B_\infty(\infty, \rho) \subset \C_\infty \setminus K$.
	\end{items}
\end{proposition}

\begin{remark}
	Parts \ref{p7.12i} and \ref{p7.12ii} together imply that the subspace topology on $\C \subset \C_\infty$ and the usual topology $(\C, \abs{})$ are the same.
	In other words, things will converge in one if and only if they converge in the other (though the speed of convergence needn't be the same).

	Parts \ref{p7.12iii} and \ref{p7.12iv} tell us that $\C_\infty$ is, in fact, a one-point compactification\index{compactification} of $\C$.
\end{remark}

Recall how a function $f \colon G \to \C$ is called \keyword{meromorphic}\index{meromorphic function} if it is analytic except for isolated poles.
Now that we have a metric on the Riemann sphere, we can view this in a different light: a meromorphic function is an analytic function (hence continuous) which sometimes reaches infinity.
In other words, $f \colon G \to \C_\infty$ is continuous.

In this light, if we let $M(G)$ denote the set of all meromorphic functions on $G$, then $M(G) \subset C(G, \C_\infty)$, where $C(G, \C_\infty)$ is a complete metric space in $d$.

\begin{remark}
	Unlike $H(G)$, the space $M(G)$ of meromorphic functions is \emph{not} complete, for it is not closed.

	For example, if we let $f_n(z) = n$ be constant functions, then $\Set{f_n}$ is a Cauchy sequence in $C(G, \infty)$ (they get close to $\infty$), and $f_n \to f$ where $f \equiv \infty$.
	But this limit is not meromorphic.

	It is instructive to note that this does not converge at all in $C(G, \C)$---convergence in $C(G, \C_\infty)$ is not quite the same.
	This example then also demonstrates that in $C(G, \C_\infty)$, $H(G)$ also stops being closed.
\end{remark}

That said, $M(G)$ is, in some sense, very close to being complete: the \emph{only} function it is missing is the one that is constantly infinity:

\begin{theorem}\label{thm7.13}
	\begin{items}
		\item Let $\Set{f_n} \subset M(G)$.
		Suppose $f_n \to f$ in $C(G, \C_\infty)$.
		Then either $f \in M(G)$ or $f \equiv \infty$.
		\item Suppose $\Set{f_n} \subset H(G)$ and $f_n \to f$ in $C(G, \C_\infty)$.
		Then either $f \in H(G)$ or $f \equiv \infty$.
	\end{items}
\end{theorem}

Consequently,

\begin{corollary}\label{cor7.14}
	$\closure{M(G)} = M(G) \cup \Set{\infty}$ is a complete metric space.
\end{corollary}

\begin{corollary}\label{cor7.15}
	$\closure{H(G)} = H(G) \cup \Set{\infty}$ is a closed in $C(G, \C_\infty)$ and hence complete.
\end{corollary}
