%!TEX root = ../lectures.tex

\lecture[September 5th, 2019]{Counting Zeros}

\topic{Simply connected sets and Simple closed curve}

\begin{proposition}[Independence of path]\label{prop4.6}
	Let $\gamma_0$ and $\gamma_1$ be two rectifiable curves in in an open set $G$, both from $a$ to $b$, and suppose $\gamma_0 \simeq \gamma_1$.
	Then
	\[
		\int_{\gamma_0} f(z) \, d z = \int_{\gamma_1} f(z) \, d z
	\]
	for any analytic function $f$ on $G$.
\end{proposition}

\begin{proof}
	Consider the path $\gamma_0 \cup - \gamma_1$, which is then a closed path from $a$ to itself.
	Then
	\[
		0 = \int_{\gamma_0 \cup - \gamma_1} f(z) \, d z = \int_{\gamma_0} f(z) \, d z - \int_{\gamma_1} f(z) \, d z. \qedhere
	\]
\end{proof}

\begin{corollary}\label{cor4.7}
	Let $G$ be simply connected and let $f \colon G \to \C$ be analytic.
	Then $f$ has a primitive.
\end{corollary}

\begin{proof}
	We use the method from our proof of \nameref{thm4.5}: since $G$ is simply connected, and $f$ is analytic, the integral of $f$ over any simple closed curve is $0$.
	So in particular the integral of $f$ over any triangular path is $0$, so we can construct a primitive of $f$ in the same way we did for \nameref{thm4.5} theorem.
\end{proof}

\begin{corollary}\label{cor4.8}
	Let $G$ be simply connected and $f \colon G \to \C$ be an analytic function.
	Suppose $f(z) \neq 0$ for all $z \in G$.
	Then there exists an analytic function $g \colon G \to \C$ such that $f(z) = \exp(g(z))$.
	Moreover, for fixed $z_0 \in G$ and $f(z_0) = e^{w_0}$, we can choose $g$ such that $g(z_0) = w_0$.
\end{corollary}

Before we go on to prove this, let us briefly discuss the idea behind this result.
If we have a nonzero thing in $\R$, we can take its logarithm.
The same thing is not \emph{quite} true in $\C$, because the complex logarithm is not analytic---it requires a branch cut, but it does make sense to take logarithms \emph{locally}, just make sure the branch cut doesn't interfere in the local neighbourhood at hand.

Hence, since we can't take logarithms globally, we instead express a sort of inverse of the result, hence the exponential function.

\begin{proof}
	Since $f(z) \neq 0$ for all $z \in G$ and $f'$ exists since $f$ is analytic, we must have that its logarithmic derivative\index{logarithmic derivative} $\frac{f'}{f}(z)$ is analytic in $G$.

	Hence by \autoref{cor4.7} $\frac{f'}{f}(z)$ has a primitive, say $g_1$.
	Setting $h(z) = \exp(g_1(z))$, this must be analytic in $G$ and moreover because of the exponential $h(z) \neq 0$ for all $z \in G$.
	Thus $\frac{f}{h}$ is analytic in $G$, and computing its derivative we get
	\[
		\Bigl ( \frac{f}{h} \Bigr )' = \frac{f' h - f h'}{h^2} = \frac{f' e^{g_1} - f e^{g_1} g_1'}{h^2} = \frac{f' e^{g_1} - f e^{g_1} \frac{f'}{f}}{h_2} = 0.
	\]
	Therefore $\frac{f}{h}$ is constant, so, say, $f = c \cdot h$ for some $c \neq 0$.
	But then
	\[
		f(z) = c \exp(g_1(z)) = \exp(g(z))
	\]
	for some scaled version $g(z)$ of $g_1(z)$.

	Note finally how, since the exponential is invariant under shifts of $2 \pi i k$ for integers $k$, this is equal to $\exp(g(z) + 2 \pi i k)$ for all integers $k$, and so if we want
	\[
		f(z_0) = e^{w_0} = \exp(g(z) + 2 \pi i k),
	\]
	we need just choose $k$ such that $w_0 = g(z_0) + 2 \pi  k$ and rename $g$ to include this shift.
\end{proof}

\topic{Counting zeros}

Recall how if $G$ is a region and $f \colon G \to \C$ is analytic with zeros $a_1, a_2, \dots, a_m$ (repeated according to multiplicities), we can write
\[
	f(z) = (z - a_1) (z - a_2) \dotsm (z - a_m) g(z)
\]
where $g$ is analytic in $G$ and $g(z) \neq 0$ for all $z \in G$.
Taking the logarithmic derivative of this we get
\[
	\frac{f'}{f}(z) = \frac{1}{z - a_1} + \frac{1}{z - a_2} + \dots + \frac{1}{z - a_m} + \frac{g'}{g}(z).
\]
Notice how $\frac{g'}{g}$ is analytic in $G$ since $g$, by construction, is never zero.
Hence if we integrate this expression over a closed rectifiable curve $\gamma \simeq 0$, the last term goes away, and importantly, by Cauchy's integral formula, each of the remaining terms will contribute exactly $2 \pi i$ times the winding number of $\gamma$ around them if they are inside the region enclosed by $\gamma$, and otherwise zero.
Therefore

\begin{theorem}\label{thm4.9}
	Let $G$ be a region and let $f$ be an analytic function on $G$ with zeros $a_1, a_2, \dots, a_m$, repeated according to multiplicities.
	Let $\gamma$ be a closed rectifiable curve in $G$ such that $a_i \not\in \Set{\gamma}$ for any $i = 1, 2, \dots, m$, and $\gamma \simeq 0$ in $G$.
	Then
	\[
		\frac{1}{2 \pi i} \int_\gamma \frac{f'}{f}(z) \, d z = \sum_{i = 1}^{m} n(\gamma; a_i).
	\]
\end{theorem}

In practice, we often care about a simpler version of this result:

\begin{corollary}
	Let $G$ be a region and let $\gamma$ be a simple closed curve in $G$.
	Then the integral
	\[
		\int_\gamma \frac{f'}{f}(z) \, d z
	\]
	counts the zeros of $f$ enclosed by $\gamma$ with multiplicities (assuming no zeros of $f$ lie on $\gamma$).
\end{corollary}

\begin{theorem}\label{thm4.10}
	Let $f$ be analytic in $B(a, R)$, and let $f(a) = \alpha$.
	Suppose $f(z) - \alpha$ has a zero of order $m$ at $z = a$.
	Then there exist $\varepsilon > 0$ and $\delta > 0$ such that for all $0 < \abs{\xi - \alpha} < \delta$, the function $f(z) - \xi$ has exactly $m$ simple roots in $B(a, \varepsilon)$.
\end{theorem}

\begin{remark}
	This theorem implies that $f(B(a, \varepsilon)) \supset B(\alpha, \delta)$.
	This will be important in just a moment.
\end{remark}

\begin{proof}
	Since the zeros of an analytic function are isolated, we can find $0 < \varepsilon < \frac{R}{2}$ such that $f(z) = \alpha$ has no solutions in $0 < \abs{z - a} < 2 \varepsilon$, and $f'(z) \neq 0$ in $0 < \abs{z - a} < 2 \varepsilon$.
	For this second part, note that it is of course possible that $f'$ has a zero near $a$, but $f'$ is also analytic, and its zeros are also isolated, so in that case shrink the ball some more.

	Let $\gamma(t) = a + \varepsilon \exp(2 \pi i t)$ for $0 \leq t \leq 1$, and let $\sigma = f \circ \gamma$, the image of this circle under $f$.

	Since $\alpha = f(a) \not\in \Set{\sigma}$, there exists some $\delta > 0$ such that $B(\alpha, \delta) \cap \Set{\sigma} = \varnothing$.
	Hence $B(\alpha, \delta) \subset \C \setminus \Set{\sigma}$.
	Therefore, since both $\alpha$ and $\xi$ are inside the region enclosed by $\sigma$,
	\[
		n(\sigma; \alpha) = \frac{1}{2 \pi i} \int_\sigma \frac{1}{z - \alpha} \, d z = \frac{1}{2 \pi i} \int_\sigma \frac{1}{z - \xi} \, d z.
	\]
	But $\sigma = f \circ \gamma$, so taking the change of variables $z = f(w) - \alpha$, $d z = (f(w) - \alpha)' \, d w$ in the first integral, we get
	\[
		\frac{1}{2 \pi i} \int_\gamma \frac{(f(w) - \alpha)'}{f(w) - \alpha} \, d w ,
	\]
	which is the number of zeros of $f(z) - \alpha$ in $B(a, \varepsilon)$, or in other words $m$.
	Doing the same thing to the second integral we get that
	\[
		\frac{1}{2 \pi i} \int_\sigma \frac{1}{z - \xi} \, d z
	\]
	is the number of zeros of $f(z) - \xi$ in $B(a, \varepsilon)$, but we we chose $\varepsilon$ such that $f'(z) \neq 0$ for all $0 < \abs{z - a} < 2 \varepsilon$, so each of these roots of $f(z) - \xi$ is simple.
\end{proof}

The remark, as promised, is fairly important:

\index{open mapping theorem|(}
\begin{theorem}[Open mapping theorem]\label{thm4.11}
	Let $G$ be a region.
	Suppose $f$ is a non-constant analytic function on $G$.
	Then for any open set $U \subset G$, $f(U)$ is open.
\end{theorem}
\index{open mapping theorem|)}

\begin{proof}
	Let $U \subset G$ be open.
	We want to show that $f(U)$ is open, i.e., for every $\alpha \in f(U)$ there exists some $\delta > 0$ such that $B(\alpha, \delta) \subset f(U)$.

	To this end, suppose $a \in G$ such that $f(a) = \alpha$.
	By the previous theorem there exists $\varepsilon > 0$ and $\delta > 0$ such that $B(a, \varepsilon) \subset U$ and $f(B(a, \varepsilon)) \supset B(\alpha, \delta)$, and $f(U) \supset f(B(a, \varepsilon))$, so $f(U)$ is open.
\end{proof}

As an aside, this implies that an injective analytic function has analytic inverse, since it says that preimages of open sets under the inverse are open sets.
Though we know more than that about an injective analytic function: its derivative must be nonzero everywhere, so in fact it would be a conformal mapping:

\begin{exercise}\label{hw3.1}
	Let $G$ be a region.
	Suppose that $f \colon G \to \C$ is analytic and one-to-one.
	Show that $f'(z) \neq 0$ for all $z \in G$.
\end{exercise}

\begin{remark}
	The converse of this exercise is not true: there exist analytic functions with non-vanishing derivatives that are not one-to-one.
	An easy example if $f(z) =e^z$; the derivative of this is never zero, but $f(z + 2 \pi i k) = f(z)$ for all integers $k$, so it is certainly not injective.
\end{remark}

We mentioned a long time ago, in \autoref{remark:analytic}~\ref{remark:analytic:1}, that requiring the derivative of a function to be continuous in the definition of analytic is superfluous.
We are finally ready to prove this:

\index{Goursat's theorem|(}
\begin{theorem}[Goursat's theorem]\label{thm4.12}
	Let $G$ be an open set and let $f \colon G \to \C$ be differentiable.
	Then $f$ is analytic.
\end{theorem}
\index{Goursat's theorem|)}

\begin{proof}
	We want to show, ultimately, that $f'$ is continuous, but it turns out a better idea, by \nameref{thm4.5}, is to show that
	\[
		\int_T f(z) \, d z = 0
	\]
	for all triangular paths $T$ in $G$, since $f$, being differentiable, is continuous, and by \nameref{thm4.5} this would make it analytic.

	Let $T$ be an arbitrary triangular path in $G$.
	Then if we form new add and subtract the line segments between the midpoints of the line segments making up $T$, as per \autoref{fig:triangularpath}, we have not changed the path, and we have
	\[
		\int_T f(z) \, d z = \sum_{i = 1}^4 \int_{T_i} f(z) \, d z.
	\]
	Now let $T^1$ be the triangular path among $T_1$, $T_2$, $T_3$, and $T_4$ such that
	\[
		\abs[\Big]{\int_{T^1} f(z) \, d z} = \max_{1 \leq i \leq 4} \abs[\Big]{\int_{T_i} f(z) \, d z}.
	\]

	\begin{marginfigure}
		\centering
		\mfincludegraphics[width=\textwidth]{figures/l6figa.tikz}

		becomes

		\mfincludegraphics[width=\textwidth]{figures/l6figb.tikz}

		\caption{\label{fig:triangularpath} Decomposing a triangular path $T$ into smaller triangular paths.}
	\end{marginfigure}

	Then by construction
	\[
		\abs[\Big]{\int_T f(z) \, d z} \leq 4 \abs[\Big]{\int_{T^1} f(z) \, d z}.
	\]
	Repeat this process with $T^1$ in place of $T$, getting $T^2$, and so on.
	In general, then, we have
	\[
		\abs[\Big]{\int_T f(z) \, d z} \leq 4^n \abs[\Big]{\int_{T^n} f(z) \, d z},
	\]
	the lengths of the paths halve every time, i.e.,
	\[
		\ell(T^n) = \frac{1}{2} \ell(T^{n - 1}) = \dots = \frac{1}{2^n} \ell(T),
	\]
	and letting $\Delta^n$ denote the closed triangle inside $T^n$, so that
	\[
		\Delta \supset \Delta^1 \supset \dots \supset \Delta^n,
	\]
	we have
	\[
		\diam(\Delta^n) = \frac{1}{2} \diam(\Delta^{n - 1}) = \dots = \frac{1}{2^n} \diam(\Delta),
	\]
	which goes to $0$ as $n \to \infty$.
	Now the intersection
	\[
		\bigcap_{n = 1}^\infty \Delta^n
	\]
	is nonempty, since the $\Delta^n$ are closed and nested, but the diameter of the intersection goes to $0$, so the intersection contains only a single point, say $z_0$.

	Since $f$ is differentiable at $z_0$, we have that for every $\varepsilon > 0$ there exists some $\delta > 0$ such that $B(z_0, \delta) \subset G$ and
	\[
		\abs[\Big]{\frac{f(z) - f(z_0)}{z - z_0} - f'(z_0)} < \varepsilon
	\]
	for every $0 < \abs{z - z_0} < \delta$.
	Hence
	\[
		\abs{f(z) - f(z_0) - f'(z_0) (z - z_0)} < \varepsilon \abs{z - z_0}.
	\]
	Now, since the functions $1$ and $z$ are analytic, by \nameref{cor4.4}
	\[
		\int_{T^n} 1 \, d z = \int_{T^n} z \, d z = 0,
	\]
	and thus for $n$ sufficiently large
	\[
		\abs[\Big]{\int_{T^n} f(z) \, d z} = \abs[\Big]{\int_{T^n} f(z) - f(z_0) - f'(z_0) (z - z_0) \, d z} \leq \varepsilon \int_{T^n} \abs{z - z_0} \, d z.
	\]
	But $z$ is on $T^n$ and $z_0$ is in the interior of $\Delta^n$, so this distance is bounded by $\diam(\Delta^n)$.
	Hence the integral is bounded above by
	\[
		\varepsilon \diam(\Delta^n) \ell(T^n) = \frac{\varepsilon}{4^n} \diam(\Delta) \ell(T).
	\]
	Hence
	\[
		\abs[\Big]{\int_T f(z) \, d z} \leq 4^n \frac{\varepsilon}{4^n} \diam(\Delta) \ell(T) = \varepsilon \diam(\Delta) \ell(T),
	\]
	but $\varepsilon > 0$ is arbitrary, so
	\[
		\abs[\Big]{\int_T f(z) \, d z} = \int_T f(z) \, d z = 0,
	\]
	as desired.
\end{proof}
