%!TEX root = ../lectures.tex

\lecture[November 5th, 2019]{Range of Analytic Functions}

\topic{Proof of Hadamard's factorisation theorem}

We start by finishing the proof started last time:

\begin{proof}[Proof of \nameref{thm8.15}, continued]
	By the \nameref{thm8.8}, we therefore have
	\[
		f(z) = z^m \exp(g(z)) \prod_{n = 1}^\infty E_p\Bigl( \frac{z}{a_n} \Bigr),
	\]
	where for ease of discussion we will call the product at the end $P(z)$.
	We next wish to show that $g(z)$ is a polynomial of degree at most $\lambda$.

	Consider the function $\frac{f(z)}{z^m P(z)}$.
	It is entire (or has only removable singularities, so remove them) and never vanishes in $\C$.
	Then, studying the orders of these terms,
	\begin{align*}
		\log \abs[\Big]{\frac{f(z)}{z^m P(z)}} &= \log \abs{f(z)} - m \log\abs{z} - \log \abs{P(z)} \\
		&\leq C_\varepsilon \left ( \abs{z}^{\lambda + \varepsilon} + \abs{z}^\varepsilon + \abs{z}^p \right ) \\
		&\leq K_\varepsilon \abs{z}^{\lambda + \varepsilon}
	\end{align*}
	for $\abs{z}$ large, since $p \leq \lambda$ and the terms in $P(z)$ are of the form
	\[
		E_p(z) = (1 - z) \exp\Bigl( z + \frac{z^2}{2} + \dots + \frac{z^p}{p} \Bigr),
	\]
	at least for $p \geq 1$ (for $p = 0$ it is just $1 - z$).

	In other words $\frac{f(z)}{z^m P(z)}$ is of finite order at most $\lambda$, and never vanishes, which by \autoref{prop8.14} means
	\[
		\frac{f(z)}{z^m P(z)} = \exp(g(z))
	\]
	where $g(z)$ is a polynomial of degree at most $\lambda$, meaning that $f(z)$ is of finite genus at most $\lambda$, finishing the proof.
\end{proof}

As a small historic interlude, this theorem gives in particular a factorisation of the \keyword{Riemann zeta function}\index{Riemann zeta function} in terms of its zeros, a product representation first conjectured to exist by Riemann.
It was in proving the existence of this product representation of the zeta function that Hadamard happened to prove the existence for all entire functions of finite order.

Aside from genus and order, there is a third related, and sometimes powerful, notion.
To see how this is occassionally more useful, see in particular \autoref{hw8p6}~\ref{hw8p6b}.

\begin{definition}[Exponent of convergence]
	Let $\Set{a_n}$ be a sequence of non-zero complex numbers.
	Let
	\[
		\rho = \inf\Set*{r \given \sum_n \frac{1}{\abs{a_n}^r} < \infty},
	\]
	called the \keyword{exponent of convergence}\index{exponent of convergence} of $\Set{a_n}$.
\end{definition}

\begin{exercise}
	\begin{parts}
		\item Let $f$ be an entire function of rank $p$.
		Show that the exponent of convergence $\rho$ of the non-zero zeros of $f$ satisfies: $p \leq \rho \leq p + 1$.

		\item Let $f$ be an entire function of order $\lambda$ and let $\Set{a_n}$ be the non-zero zeros of $f$ repeated according to multiplicity.
		Let $\rho$ be the exponent of convergence of $\Set{a_n}$.
		Show that $\rho \leq \lambda$.

		\item Let $\displaystyle P(z) = \prod_{n = 1}^\infty E_p\Bigl( \frac{z}{a_n} \Bigr)$ be a canonical product of rank $p$.
		Let $\rho$ be the exponent of convergence of $\Set{a_n}$.
		Show that the order of $P(z)$ is $\rho$. \qedhere
	\end{parts}
\end{exercise}

\begin{exercise}\label{hw8p6}
	\begin{parts}
		\item\label{hw8p6a} Let $f$ and $g$ be entire functions of finite order $\lambda$.
		Suppose that $f(a_n) = g(a_n)$ for $\Set{a_n}$ such that $\displaystyle \sum_n \abs{a_n}^{-(\lambda + 1)} = \infty$.
		Show that $f = g$.

		\item\label{hw8p6b} Replace the condition in \ref{hw8p6a} by $\displaystyle \sum_n \abs{a_n}^{-(\lambda + \varepsilon)} = \infty$ for some $\varepsilon > 0$.
		Show that $f = g$.

		\item Find all entire functions $f$ of finite order such that $f(\log n) = n$ for $n = 1, 2, \dots$. \qedhere
	\end{parts}
\end{exercise}

\topic{The range of entire functions}

Consider a polynomial $p(z)$, say of degree $n$, and let $c \in \C$.
Then the polynomial $p(z) - c$ is also of degree $n$, and hence has precisely $n$ roots (counting multiplicity), according to the \nameref{thm3.10}.
This means that the range of a polynomial is all of $\C$, and more precisely every point $c \in \C$ is attained exactly $n$ times.

We wish to investigate the same question, that of the range, for entire functions in general.

\index{Picard's theorem!special case|(}
\begin{theorem}[Special case of Picard's theorem]\label{thm8.16}
	Let $f$ be a non-constant entire function of finite order.
	Then $f(z)$ assumes each complex number with only one possible exception.
\end{theorem}
\index{Picard's theorem!special case|)}

\begin{proof}
	Suppose there exist $\alpha, \beta \in \C$ be distinct points such that $f(z) \neq \alpha$ and $f(z) \neq \beta$ for all $z \in \C$.
	Then $f(z) - \alpha$ is entire and never vanishes in $\C$, so by \autoref{prop8.14} we can write $f(z) - \alpha = \exp(g(z))$ where $g(z)$ is a polynomial.
	That is to say, $f(z) = \exp(g(z)) + \alpha$.

	Similarly, since $f(z) \neq \beta$ for all $z \in \C$ we have $\exp(g(z)) + \alpha \neq \beta$ for all $z \in \C$, meaning that $\exp(g(z)) \neq \beta - \alpha$.

	Hence $g(z) \neq \log(\beta - \alpha)$ (note how $\beta - \alpha \neq 0$).
	But $g(z)$ is a polynomial, so its range is all of $\C$, making this a contradiction.
\end{proof}

This means that an entire function can miss only one point in $\C$.

The polynomial example brings up another natural question: if $f$ is an entire function, how many times can $f(z)$ assume $a \in \C$?

As discussed, in the case of a degree $n$ polynomial, the answer is precisely $n$ times.
We answer the question for entire functions analogously:

\begin{theorem}\label{thm8.17}
	Let $f$ be an entire function of finite order $\lambda$, where $\lambda$ is not an integer.
	Then $f$ has infinitely many zeros.
\end{theorem}

Note that the order not being integral implies $f$ is non-constant, since (nonzero) constants are of order $0$.

\begin{proof}
	Suppose $f$ has only finitely many zeros, say $\Set{a_1, a_2, \dots, a_n}$, repeated according to multiplicity.
	Then we can write
	\[
		f(z) = (z - a_1) (z - a_2) \dotsm (z - a_n) \exp(g(z)),
	\]
	with $g$ entire.
	Consider
	\[
		\log \abs[\Big]{\frac{f(z)}{(z - a_1) \dotsm (z - a_n)}} = \log \abs{f(z)} - \log\abs{z - a_1} - \dots - \log\abs{z - a_n} \leq C_\varepsilon \abs{z}^{\lambda + \varepsilon}
	\]
	for $\abs{z}$ large.
	Hence $\frac{f(z)}{(z - a_1) \dotsm (z - a_n)}$ has order at most $\lambda$, so $g(z)$ is a polynomial of degree $\lambda$ by \autoref{prop8.14}.
	Hence the order of $f$ is the degree of $g$, which, being a polynomial, is an integer.
	This is a contradiction.
\end{proof}

It is worth noting that the key insight in the above proof is that scaling by polynomials don't affect the order of an entire function.

\begin{corollary}\label{cor8.18}
	Let $f$ be an entire function of order $\lambda \not\in \Z$.
	Then $f$ assumes each complex number an infinite number of times (hence there are no exceptional points).
\end{corollary}

\begin{proof}
	Consider $g(z) = f(z) - \alpha$, $\alpha \in \C$.
	The order of $g$ is the same as the order of $f$, i.e., $\lambda \not\in \Z$.
	Hence by \autoref{thm8.17}, $g(z)$ has infinitely many zeros, meaning that $f(z)$ assumes $\alpha$ infinitely many times.
\end{proof}

That is to say, order $\lambda \not\in \Z$ guarantees the presence of the infinite product part of the \nameref{thm8.8}.

The question of the range of analytic functions gets much more delicate if we move from all of $\C$ (so entire functions) to regions $G \neq \C$.

\topic{The range of an analytic function}

Let $D = \Set{z \given \abs{z} < 1}$, and suppose $f \colon D \to \C$ is analytic, with $f(0) = 0$ and $f'(0) = 1$.
(By the \nameref{thm7.17} we know any region $G \neq \C$ is conformal to $D$, so if suffices to study $D$.)

We want to investigate how `big' $f(D)$ can be,

\begin{lemma}\label{lem9.1}
	Let $f \colon D \to \C$ be analytic, $f(0) = 0$ and $f'(0) = 1$.
	Suppose $\abs{f(z)} \leq M$ for every $z \in D$.
	Then $M \geq 1$ and $f(D) \supset B(0, \frac{1}{6 M})$.
\end{lemma}

\begin{proof}
	First we show that $M \geq 1$.
	Consider the power series expansion of $f(z)$ at $z = 0$,
	\[
		f(z) = z + a_2 z^2 + a_3 z^3 + \dots,
	\]
	since $a_0 = f(0) = 0$ and $a_1 = f'(0) = 1$ by hypothesis.
	But $0 < r < 1$, \nameref{thm4.2} gives us $\abs{a_n} \leq \frac{M}{r^n} \leq M$ for all $n$ by letting $r \to 1$.

	Hence in particular $a_1 = f'(0) = 1$, so $1 \leq \frac{M}{r}$, or $r \leq M$, for $0 < r < 1$, so letting $r \to 1$ we see that $M \geq 1$.

	Next let us show that $f(D) \supset B(0, \frac{1}{6 M})$.
	For any $w \in B(0, \frac{1}{6 M})$, consider the function $g(z) = f(z) - w$.
	We wish to show that $g(z)$ has a zero in $D$, since that corresponds to $w \in f(D)$.

	The idea is to apply \nameref{thm5.8} on the circle $\abs{z} = \frac{1}{4 M}$.
	For $\abs{z} = \frac{1}{4 M}$,
	\[
		\abs{f(z)} \geq \abs{z} - \sum_{n = 2}^\infty \abs{a_n} \abs{z}^n \geq \frac{1}{4 m} - \sum_{n = 2}^\infty M \Bigl( \frac{1}{4 M} \Bigr)^n
	\]
	since $\abs{a_n} \leq M$ for all $n$.
	The sum at the end is geometric, and works out to be $\frac{1}{16M - 4}$, so
	\[
		\abs{f(z)} \geq \frac{1}{4 M} - \frac{1}{16 M - 4} \geq \frac{1}{6 M}.
	\]
	Hence on $\abs{z} = \frac{1}{4 M}$, since $w \in B(0, \frac{1}{6 M})$,
	\[
		\abs{f(z) - g(z)} = \abs{w} < \frac{1}{6 M} \leq \abs{f(z)}.
	\]
	Hence by \nameref{thm5.8} $f(z)$ and $g(z)$ have the same number of zeros in $B(0, \frac{1}{4 M})$, and since $f(0) = 0$, we must consequently have $g(z) = 0$ for some $z \in B(0, \frac{1}{4 M}) \subset D$, whence $w \in f(D)$ as claimed.
\end{proof}

\begin{remark}
	It is useful to note that the appearance of the constant $6$ seems a bit out of the blue, and it sort of is.

	As is often the case in situations like this, the way one originally discovers the theorem is to consider some constant $C$ in place of $6$, prove the theorem in some generality, and then, once done, optimise the choice of $C$.

	The appearance of $6$ only looks vaguely magical because the exposition here skips the hard work in finding it.
\end{remark}

Heuristically, we can tell this lemma cannot be optimal: when $M$ is large, i.e., we have a large bound on $f(z)$, so $f$ takes on many values between $0$ and $M$, then $B(0, \frac{1}{6 M})$ is tiny.
This hints at there being a better result, and indeed we will work on achieving it in the near future.

For reference we also work out what this theorem becomes on arbitrary balls centred at $z = 0$:

\begin{lemma}\label{lem9.2}
	Suppose $g$ is analytic on $B(0, R)$, $g(0) = 0$, and $\abs{g'(0)} = \mu > 0$ and $\abs{g(z)} \leq M$ for all $z \in B(0, R)$.
	Then $g(B(0, R)) \supset B(0, \frac{R^2 \mu^2}{6 M})$.
\end{lemma}

\begin{proof}
	Consider the function $f(z) = \frac{g(R z)}{R g'(0)} \colon D \to \C$.
	We verify that $f(0) = 0$,
	\[
		f'(0) = \frac{g'(R z) R}{R g'(0)} \biggr\rvert_{z = 0} = \frac{g'(0) R}{R g'(0)} = 1,
	\]
	and $\abs{f(z)} \leq \frac{M}{R \mu}$.
	Hence \autoref{lem9.1} we get
	\[
		f(D) \supset B\Bigl( 0, \frac{1}{6 (\frac{M}{R \mu})} \Bigr) = B\Bigl( 0, \frac{R \mu}{6 M} \Bigr),
	\]
	so
	\[
		g(B(0, R)) \supset B\Bigl( 0, \frac{R^2 \mu^2}{6 M} \Bigr). \qedhere
	\]
\end{proof}

We need one more lemma in order to prove the result we are really after.

\begin{lemma}\label{lem9.3}
	Let $f$ be analytic on $B(a, r)$ such that $\abs{f'(z) - f'(a)} < \abs{f'(a)}$ for all $z \in B(a, r)$, $z \neq a$.
	Then $f$ is one-to-one.
\end{lemma}

\begin{proof}
	This is essentially the fundamental theorem of calculus.
	For $z_1, z_2 \in B(a, r)$, $z_1 \neq z_2$, write
	\[
		\abs{f(z_2) - f(z_1)} - \abs[\Big]{\int_{\interval{z_1}{z_2}} f'(z) \, d z},
	\]
	where by $\interval{z_1}{z_2}$ we mean the line segment joining $z_1$ and $z_2$ (since the ball $B(a, r)$ is convex, this line segment is in the ball).
	Adding and subtracting $f'(z)$ inside the integral, we get
	\begin{align*}
		\abs{f(z_2) - f(z_1)} &= \abs[\Big]{\int_{\interval{z_1}{z_2}} f'(z) - f'(a) + f'(a) \, d z} \\
		&\geq \abs[\Big]{\int_{\interval{z_1}{z_2}} f'(a) \, d z} - \abs[\Big]{\int_{\interval{z_1}{z_2}} f'(z) - f'(a) \, d z} \\
		&> \abs{f'(a)} \abs{z_2 - z_1} - \abs[\Big]{\int_{\interval{z_1}{z_2}} f'(a) \, d z} \\
		&= \abs{f'(a)} \abs{z_2 - z_1} - \abs{f'(a)} \abs{z_2 - z_1} = 0.
	\end{align*}
	Hence $f(z_2) \neq f(z_1)$, so $f$ is one-to-one.
\end{proof}

With these lemmata we are equipped to prove the main theorem we are after, the proof of which is very technical and little lengthy:

\index{Bloch's theorem|(}
\begin{theorem}[Bloch's theorem]\label{thm9.4}
	Let $f$ be an analytic function on a neighbourhood of $\closure{D} = \Set{z \given \abs{z} \leq 1}$ such that $f(0) = 0$ and $f'(0) = 1$.
	Then there exists a disk $S \subset D$ such that $f$ is one-to-one on $S$ and $f(S)$ contains a disk of radius $\frac{1}{72}$.
\end{theorem}

\begin{remark}
	Note how this constant $\frac{1}{72}$, quite remarkably, is uniform in $f$.
\end{remark}
