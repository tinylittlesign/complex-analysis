%!TEX root = ../lectures.tex

\lecture[October 31st, 2019]{Order}

\topic{The order of entire functions}

We showed last time that if $f$ is an entire function of finite genus $\mu$, then $f(z) \ll \exp(\abs{z}^{\mu + 1 + \varepsilon})$ for any $\varepsilon > 0$.

\begin{definition}[Order]
	An entire function $f$ is of \keyword{finite order}\index{order!finite} if there exists $a \geq 0$ such that $f(z) \ll \exp(\abs{z}^a)$.

	If $f$ is not of finite order, then we say $f$ is of \keyword{infinite order}\index{order!infinite}.

	If $f$ is of finite order, then $\lambda \coloneqq \inf\Set{a \given f(z) \ll \exp(\abs{z}^a)}$ is called the \keyword{order}\index{order} of $f$.
\end{definition}

That is to say, if $f$ is of order $\lambda$, then for any $\varepsilon > 0$ we have $f(z) \ll \exp(\abs{z}^{\lambda + \varepsilon})$.

An equivalent way of saying this is that $f$ being of order $\lambda$ means that $\log \abs{f(z)} \leq C_\varepsilon \abs{z}^{\lambda + \varepsilon}$ for $\abs{z}$ large.
This is slightly delicate: when taking logarithms of $f(z) \ll \exp(\abs{z}^{\lambda + \varepsilon})$ we cannot keep the $\ll$ sign, for $f(z)$ could be very large in the negative direction, making $\abs{f(z)}$ large, reversing the inequality.

\begin{exercise}
	Look back at \autoref{hw8.2} and use it to prove that if $f$ is of finite order $\lambda$, then $n(r) \ll_\varepsilon r^{\lambda + \varepsilon}$.
\end{exercise}

\begin{exercise}
	Let $f_1$ and $f_2$ be entire functions of finite order $\lambda_1$ and $\lambda_2$ respectively.
	Let $f = f_1 + f_2$.
	\begin{parts}
		\item Show that $f$ has finite order $\lambda \leq \max\Set{\lambda_1, \lambda_2}$.
		\item Show that $\lambda = \max\Set{\lambda_1, \lambda_2}$ if $\lambda_1 \neq \lambda_2$.
		\item Give an example where $\lambda < \max\Set{\lambda_1, \lambda_2}$ with $f \neq 0$. \qedhere
	\end{parts}
\end{exercise}

\begin{exercise}
	Let $f_1$ and $f_2$ be entire functions of finite order $\lambda_1$ and $\lambda_2$ respectively.
	Let $f = f_1 f_2$.
	Show that $f$ has finite order $\lambda \leq \max\Set{\lambda_1, \lambda_2}$.
\end{exercise}

By last lecture's discussion we then also have that finite genus implies finite order:

\begin{corollary}\label{cor8.13}
	If $f$ is entire of finite genus $\mu$, then $f$ is of finite order $\lambda \leq \mu + 1$.
\end{corollary}

This means that the genus of an entire function, which by definition controls the factorisation, in fact also controls the growth of the function.
The converse is also true.

\topic{Hadamard's factorisation theorem}

\begin{proposition}\label{prop8.14}
	Let $f$ be an entire function of order $\lambda$ which does not vanish in $\C$.
	Then $f(z) = \exp(g(z))$ where $g(z)$ is a polynomial of degree $\deg g \leq \lambda$.
	Thus $f$ is of integral order $\deg g$.
\end{proposition}

\begin{proof}
	By the \nameref{thm8.8}, $f(z) = \exp(g(z))$ with $g(z)$ being entire (since $f$ has no zeros, the other two parts of the Weierstrass factorisation disappear).
	Hence let
	\[
		g(z) = \sum_{n = 0}^\infty a_n z^n
	\]
	be the power series expansion of $g$ at $z = 0$.
	In order to show that $g(z)$ is a polynomial of degree at most $\lambda$ we therefore need to show that $a_n = 0$ for all $n > \lambda$.

	Since $f$ is of order $\lambda$, for any $\varepsilon > 0$ there exists some $C_\varepsilon > 0$ such that $\Re g(z) = \log \abs{f(z)} \leq C_\varepsilon \abs{z}^{\lambda + \varepsilon}$ for $\abs{z}$ large.

	\begin{remark}
		It is instructive, at this point, to note that this is not quite as easy as invoking \autoref{ex:polynomial}, for the order only gives us a bound on $\log \abs{f(z)} = \Re g(z) \leq C_\varepsilon \abs{z}^{\lambda + \varepsilon}$, not $\abs{g(z)}$.
	\end{remark}

	Writing $\C \ni a_n = \alpha_n + i \beta_n$, where $\alpha_n, \beta_n \in \R$, and setting $z = R e^{2 \pi i \theta}$, we have from the power series
	\[
		\Re g(z) = \sum_{n = 0}^\infty \alpha_n R^n \cos(2 \pi n \theta) - \sum_{n = 1}^\infty \beta_n R^n \sin(2 \pi n \theta).
	\]
	Note that the second sum starts at $n = 1$ since $\sin(0) = 0$.

	Now we can use the orthogonality of sine and cosine to extract $\alpha_n$ and $\beta_n$.
	In particular, multiplying by $\cos(2 \pi n \theta)$ and integrating from $0$ to $1$, we extract
	\[
		\alpha_0 = \int_0^1 \Re g(R e^{2 \pi i \theta}) \, d \theta
	\]
	and
	\[
		\alpha_n R^n = 2 \int_0^1 \Re g(R e^{2 \pi i \theta}) \cos(2 \pi n \theta) \, d \theta
	\]
	for $n \geq 1$, and multiplying by $\sin(2 \pi n \theta)$ and integrating from $0$ to $1$ produces
	\[
		\beta_n R^n = 2 \int_0^1 \Re g(R e^{2 \pi i \theta}) \sin(2 \pi n \theta) \, d \theta
	\]
	for $n \geq 1$.
	Since we want to show that $a_n = 0$ for $n > \lambda$, we need to show that $\alpha_n = \beta_n = 0$ for $n > \lambda$.

	Since $\abs{\cos(2 \pi n \theta)} \leq 1$, we have
	\[
		\abs{\alpha_n R^n} \leq 2 \int_0^1 \abs{\Re g(R e^{2 \pi i \theta})} \, d \theta.
	\]
	As per the remark, we couldn't use the bound from the order directly, since the inside of the absolute value might be large in the negative direction.
	The trick here is to add and subtract $\Re g(R e^{2 \pi i \theta})$ in the integrand, resulting in
	\[
		\abs{\alpha_n R^n} \leq 2 \int_0^1 \abs{\Re g(R e^{2 \pi i \theta})} + \Re g(R e^{2 \pi i \theta}) \, d \theta - 2 \int_0^1 \Re g(R e^{2 \pi i \theta}) \, d \theta.
	\]
	The second integral is nothing but $\alpha_0 = a_0$, and the first integrand is $0$ if the real part is negative, and else just twice times itself, so
	\[
		\abs{\alpha_n R^n} \leq 2 \int_0^1 \max\Set{0, 2 C_\varepsilon R^{\lambda + \varepsilon}} \, d \theta - 2 a_0 \leq 4 C_\varepsilon R^{\lambda + \varepsilon} - 2 a_0
	\]
	since $\Re g(R e^{2 \pi i \theta}) \leq C_\varepsilon R^{\lambda + \varepsilon}$ because $f$ is of order $\lambda$.

	Hence
	\[
		\abs{\alpha_n} \leq 4 C_\varepsilon R^{\lambda - n + \varepsilon} - \frac{2 a_0}{R^n},
	\]
	which goes to $0$ as $R \to \infty$ if $n > \lambda$.
	Hence $\alpha_n = 0$ for $n > \lambda$.

	In precisely the same way, we see that $\beta_n = 0$ for $n > \lambda$, whence $a_n = \alpha_n + \beta_n = 0$ for $n > \lambda$.
	Hence $g(z)$ is a polynomial of degree at most $\lambda$, meaning that $f(z) = \exp(g(z))$ is of integral order $\deg g$.
\end{proof}

\index{Hadamard's factorisation theorem|(}
\begin{theorem}[Hadamard's factorisation theorem]\label{thm8.15}
	Let $f$ be an entire function of finite order $\lambda$.
	Then $f$ has finite genus $\mu \leq \lambda$.
\end{theorem}
\index{Hadamard's factorisation theorem|)}

In other words, the growth of an entire function also controls its factorisation.

\begin{proof}
	Let $p$ be the integer such that $p \leq \lambda < p + 1$, i.e., the integer part of $\lambda$.
	We claim that $f$ has rank at most $p$.

	To show this, let $\Set{a_n}$ be the nonzero zeros of $f$, repeated according to multiplicity, and arranged such that $0 < \abs{a_1} \leq \abs{a_2} \leq \dots$.
	To show that $f$ has rank at most $p$ we need to show that
	\[
		\sum_{n = 1}^\infty \frac{1}{\abs{a_n}^{p + 1}} < \infty.
	\]

	First, without loss of generality we may assume $f(0) = 1$, because if $f$ has a zero of order $m$ at $z = 0$, then $\frac{f(z)}{z^m}$ does not vanish at $z = 0$, and, for $z \neq 0$,
	\[
		\log \abs[\Big]{\frac{f(z)}{z^m}} = \log \abs{f(z)} - m \log\abs{z} \leq C_\varepsilon \abs{z}^{\lambda + \varepsilon},
	\]
	since $f$ is of order $\lambda$ and $\log \abs{z} \ll \abs{z}^\varepsilon$.
	Note that this estimate also holds for $z$ close to $a_n$, because then $\log \abs{f(z)}$ is large negative, so the left-hand side is even smaller.
	Hence $\frac{f(z)}{z^m}$ has the same order as $f(z)$, and since it does not affect the rank since it touches only the first part of the Weierstrass factorisation.

	Let $n(r)$ be the number of zeros of $f$ in $B(0, r)$ counted according to multiplicity, and let
	\[
		M(r) = \max_{\abs{z} = r} \abs{f(z)}.
	\]
	Then, by \autoref{hw8.2},
	\[
		n(r) \leq \frac{\log M(2 r)}{\log 2}.
	\]
	Since $f$ has order $\lambda$, for any $\varepsilon > 0$, $\log M(2 r) \ll r^{\lambda + \varepsilon}$, and consequently $n(r) \ll r^{\lambda + \varepsilon}$.

	Now having arranged $0 < \abs{a_1} \leq \abs{a_2} \leq \dots \leq \abs{a_k} \leq \dots,$ we have for any $\delta > 0$ that $n(\abs{a_k} + \delta) \geq k$.
	On the order hand, $n(\abs{a_k} + \delta) \leq C (\abs{a_k} + \delta)^{\lambda + \varepsilon}$ for some constant $C > 0$.
	Letting $\delta \to 0$, this implies $k \leq C \abs{a_k}^{\lambda + \varepsilon}$.

	Therefore
	\[
		\frac{1}{\abs{a_k}^{p + 1}} \leq k^{- \frac{p + 1}{\lambda + \varepsilon}},
	\]
	and since $p \leq \lambda < p + 1$ we can choose $\varepsilon$ small enough so that this power $\frac{p + 1}{\lambda + \varepsilon}$ is bigger than $1$.
	So for some $\varepsilon' > 0$,
	\[
		\sum_{n = 1}^\infty \frac{1}{\abs{a_n}^{p + 1}} \leq \sum_{n = 1}^\infty n^{-(1 + \varepsilon')}
	\]
	which is convergent.
	Hence $f$ has finite rank at most $p \leq \lambda$. \renewcommand{\qedsymbol}{}
\end{proof}
