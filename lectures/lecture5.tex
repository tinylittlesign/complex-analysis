%!TEX root = ../lectures.tex

\lecture[September 3rd, 2019]{Cauchy's Integral Theorem}

\topic{Winding numbers}

Let $\gamma_1(t) = a + r e^{i t}$ for $0 \leq t \leq 2 \pi$, i.e., the anticlockwise circle of radius $r$ around $a$.
Then
\[
	\frac{1}{2 \pi i} \int_{\gamma_1} \frac{1}{z - a} \, d z = \frac{1}{2 \pi i} \int_0^{2 \pi} \frac{i r e^{i t}}{a + r e^{i t} - a} \, d t = \frac{1}{2 \pi} \int_0^{2 \pi} \, d t = 1.
\]
On the other hand, for $b$ outside the region enclosed by $\gamma_1$, we get
\[
	\frac{1}{2 \pi i} \int_{\gamma_1} \frac{1}{z - b} \, d z = 0,
\]
since the integrand is analytic on the region enclosed by $\gamma_1$.

Similarly, consider another curve $\gamma_2(t) = a + r e^{i t}$, but this time for $0 \leq t \leq 4 \pi$.
Then
\[
	\frac{1}{2 \pi i} \int_{\gamma_2} \frac{1}{z - a} \, d z = 2.
\]

So it appears, in general, as though this kind of integral counts how many times the curve goes around $a$.
Indeed this is the case, and first let us make sure it counts in a reasonable way:

\begin{proposition}\label{prop4.1}
	If $\gamma \colon \interval{0}{1} \to \C$ is a closed rectifiable curve and $a \not\in \Set{\gamma}$.
	Then
	\[
		\frac{1}{2 \pi i} \int_\gamma \frac{1}{z - a} \, d z
	\]
	is an integer.
\end{proposition}

\begin{proof}
	We may assume that $\gamma$ is a smooth curve (otherwise approximate it by a smooth curve).
	Define the function
	\[
		g(t) = \int_0^t \frac{\gamma'(s)}{\gamma(s) - a} \, d s,
	\]
	so that $g(0) = 0$ and
	\[
		g(1) = \int_\gamma \frac{1}{z - a} \, d z.
	\]
	The strategy is this: since $e^{2 \pi i k} = 1$ for all integers $k$, we want to show that something like $e^g(t)$ is constant.
	To this end, note how by the Fundamental theorem of calculus
	\[
		g'(t) = \frac{\gamma'(t)}{\gamma(t) - a},
	\]
	and notice how
	\[
		\frac{d}{d t} \left ( e^{-g(t)} (\gamma(t) - a) \right ) = e^{-g(t)} \gamma'(t) - g'(t) e^{-g(t)} (\gamma(t) - a) = 0.
	\]
	Hence $e^{-g(t)} (\gamma(t) -a) = e^{-g(0)} (\gamma(0) - a) = \gamma(0) - a$ is a constant, since $g(0) = 0$.
	But $\gamma$ is a closed curve, so $\gamma(0) = \gamma(1)$, meaning that
	\[
		e^{-g(1)} (\gamma(1) - a) = \gamma(0) - a,
	\]
	meaning that $e^{-g(1)} = 1$, so $g(1) = 2 \pi i k$ for some integer $k$.
\end{proof}

Consequently we make the following definition.

\begin{definition}[Winding number]
	If $\gamma$ is a closed rectifiable curve in $\C$, then for $a \not\in \Set{\gamma}$,
	\[
		n(\gamma; a) \coloneqq \frac{1}{2 \pi i} \int_\gamma \frac{1}{ z - a} \, d z \in \Z
	\]
	is called the \keyword{winding number}\index{winding number} of $\gamma$ around $a$.
	This simply counts how many times the curve $\gamma$ winds around $a$, as suggested by the name.
\end{definition}

\begin{example}
	If $a$ belongs to the unbounded component of $\C \setminus \Set{\gamma}$ for some closed rectifiable curve $\gamma$, then of course $n(\gamma; a) = 0$, since $\frac{1}{z - a}$ is analytic there.
\end{example}

\index{Cauchy's integral formula|(}
\begin{theorem}[Cauchy's integral formula]\label{thm4.2}
	Let $G$ be an open set and $f \colon G \to \C$ be analytic.
	Suppose $\gamma$ is a closed rectifiable curve in $G$ such that $n(\gamma; w) = 0$ for all $w \in \C \setminus G$.
	Then for $a \in G \setminus \Set{\gamma}$,
	\[
		n(\gamma; a) f(a) = \frac{1}{2 \pi i} \int_\gamma \frac{f(z)}{z - a} \, d z.
	\]
\end{theorem}
\index{Cauchy's integral formula|)}

\begin{remark}
	The condition $n(\gamma; w) = 0$ for all $w \in \C \setminus G$ is to avoid cases where, for instance, $G$ is an annulus and with $w$ in the inside the inner disk and $\gamma$ around it.

	Often in practice we use this theorem for simply connected $G$, where this is not a problem.
\end{remark}

\begin{corollary}\label{cor4.3}
	Let $G$ be an open set and $f \colon G \to \C$ be analytic.
	Suppose $\gamma$ is a closed and rectifiable curve in $G$ such that $n(\gamma; w) = 0$ for all $w \in \C \setminus G$.
	Then for $a \in G \setminus \Set{\gamma}$,
	\[
		n(\gamma; a) f^{(k)} (a) = \frac{k!}{2 \pi i} \int_\gamma \frac{f(z)}{(z - a)^{k + 1}} \, d z.
	\]
\end{corollary}

\index{Cauchy's theorem|(}
\begin{corollary}[Cauchy's theorem]\label{cor4.4}
	Let $G$ be an open set and $f \colon G \to \C$ an analytic function.
	Suppose $\gamma$ is a closed and rectifiable curve in $G$ such that $n(\gamma, w) = 0$ for all $w \in \C \setminus G$.
	Then
	\[
		\int_\gamma f(z) \, d z = 0.
	\]
\end{corollary}
\index{Cauchy's theorem|)}

\begin{proof}
	Replace $f(z)$ by $f(z) (z - a)$ in Cauchy's integral formula, making the left-hand side $0$ and the right-hand side the above integral.
\end{proof}

It is useful to note that all of these, along with Cauchy's residue theorem (which we might talk about in future) are equivalent.

An interesting question one might naturally ask is if there are other functions that satisfy
\[
	\int_\gamma f(z) \, d z = 0
\]
for all closed curves $\gamma$.
The answer is no:

\index{Morera's theorem|(}
\begin{theorem}[Morera's theorem]\label{thm4.5}
	Let $G$ be a region and let $f \colon G \to \C$ be continuous such that
	\[
		\int_T f(z) = 0
	\]
	for all triangular paths\index{triangular path} $T$ in $G$.\footnote{By triangular path we mean simply a curve composed of three line segments meeting end on end.}
	Then $f$ is analytic in $G$.
\end{theorem}
\index{Morera's theorem|)}

\begin{proof}
	We will show that $f$ has a primitive $F$, i.e., $F'(z) = f(z)$ in $G$.
	This implies that $F$ and $f$ are both analytic, since one derivative implies infinite derivatives in $\C$.

	Without loss of generality we may assume that $G = B(a, R)$, since if we can make the theorem work on any small ball in $G$, then we can make it work in all of $G$.

	For $z \in G$, define
	\[
		F(z) = \int_{\interval{a}{z}} f(z) \, d z,
	\]
	where by $\interval{a}{z}$ we mean the line segment joining $a$ and $z$, with that orientation.
	Then for any $z_0 \in G$ we have
	\[
		F(z) = \int_{\interval{a}{z}} f(z) \, d z = \int_{\interval{a}{z_0}} f(z) \, d z + \int_{\interval{z_0}{z}} f(z) \, d z
	\]
	since the interval over all three segments added together is $0$ by hypothesis.
	Therefore
	\[
		\frac{F(z) - F(z_0)}{z - z_0} = \frac{\int_{\interval{z_0}{z}} f(z) \, d z}{z - z_0}.
	\]
	We expect this to be $f(z_0)$, so we study the difference
	\[
		\abs[\Big]{\frac{F(z) - F(z_0)}{z - z_0} - f(z_0)} = \abs[\Big]{ \frac{1}{z - z_0} \int_{\interval{z_0}{z}} (f(w) - f(z_0)) \, d w}.
	\]
	We can bound the integrand by its supremum on $\interval{z_0}{z}$, so this is bounded by
	\[
		\frac{1}{\abs{z - z_0}} \int_{\interval{z_0}{z}} \sup_{w \in \interval{z_0}{z}} \abs{f(w) - f(z_0)} \, d w = \sup_{w \in \interval{z_0}{z}} \abs{f(w) - f(z_0)}
	\]
	since the integrand no longer depends on $w$.
	Taking the limit as $z \to z_0$, this goes to $0$ since $f$ is continuous, and hence $F'(z_0) = f(z_0)$ for every $z_0 \in G$.
\end{proof}

\topic{Homotopy}

\begin{definition}[Homotopic]
	We say that two curves $\gamma_1$ and $\gamma_2$ are \keyword{homotopic}\index{homotopic} in a domain $G$ if they can be continuously deformed from one to the other inside of the domain.
\end{definition}

\begin{definition}[Simply connected]
	We say that a set $G$ is \keyword{simply connected}\index{simply connected} if every closed curve $\gamma$ in $G$ is homotopic to a point, denoted $\gamma \simeq 0$.
\end{definition}
