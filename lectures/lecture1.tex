%!TEX root = ../lectures.tex

\lecture[August 20th, 2019]{Review of Basic Complex Analysis}

\topic{Basics}

Throughout we will denote complex numbers by $z = x + i y$, where $x, y \in \R$ and $i = \sqrt{-1}$.
We call $x = \Re(z)$ and $y = \Im(y)$, the \keyword{real}\index{real part} and \keyword{imaginary}\index{imaginary part} parts of $z$, respectively.
Moreover we call $\conjugate{z} = x - i y$ the \keyword{complex conjugate}\index{complex conjugate} of $z$ and $\abs{z} = \sqrt{x^2 + y^2} = \abs{\conjugate{z}}$ the \keyword{absolute value} or \keyword{modulus}\index{absolute value}\index{modulus|see {absolute value}} of $z$.
Note, which occasionally comes in handy, that $\abs{z}^2 = z \conjugate{z}$.

Also commonplace is to discuss \keyword{polar representation}\index{polar form} of complex numbers.
To this end, let $\theta \in \R$ and define
\[
	e^{i \theta} \coloneqq \cos(\theta) + i \sin(\theta).
\]
This way $e^{i \theta}$ is a point on the unit circle in the complex plane, and so naturally we can scale this by a real number to reach any point in the plane.

Since the complex plane $\C$ has an absolute value $\abs{}$, as per above, we also have an induced metric, namely for $z_1, z_2 \in \C$ we define the distance $d(z_1, z_2) = \abs{z_1 - z_2}$.
Under this metric $(\C, d)$ is a complete metric space.

We will take a moment to discuss some commonly useful functions on the complex plane.
The first suspect we have almost seen above already, namely $e^z$.
If $z = x + i y$, then of course
\[
	e^z = e^{x + i y} = e^x \cdot e^{i y} = e^x (\cos(y) + i \sin(y)),
\]
so $\abs{e^z} = e^x = e^{\Re(z)}$.

Close cousin of the exponential function are the basic trigonometric functions; we would like for $\sin(z)$ and $\cos(z)$ to satisfy $e^z = \cos(z) + i \sin(z)$ (and, correspondingly, $e^{i z} = \cos(z) - i \sin(z)$), and so to that end we define
\[
	\cos(z) \coloneqq \frac{e^{i z} + e^{- i z}}{2}
\]
and
\[
	\sin(z) \coloneqq \frac{e^{i z} - e^{- i z}}{2 i}.
\]

Another close cousin to the exponential function is the natural logarithm.
This, for the first time in our brief exploration, is where things get a little bit hairy.
We of course would like for $e^{\log (z)} = z$, since we want $\log$ to be the inverse function of the exponential.
Now, writing $z = \abs{z} e^{i \theta}$, where $\theta = \arg(z)$, its argument, we can try to puzzle out what $\log(z)$ should look like, by setting $\log(z) = u + i v$ and studying $e^{\log(z)}$.

Since we then have
\[
	e^{\log(z)} = e^{u + i v} = e^u \cdot e^{i v},
\]
and in turn we want $e^{\log(z)} = z = \abs{z} e^{i \theta}$ we have, comparing sides, $e^u = \abs{z}$, i.e., $u = \log\abs{z}$, and $e^{i v} = e^{i \theta}$.
This last one is where it gets hairy: this does not imply $v = \theta$, but more loosely that $v = \theta + k 2 \pi$ for any $k \in \Z$.

Thus
\[
	\log(z) \coloneqq \log\abs{z} + i(\theta + k 2 \pi)
\]
for $k \in \Z$, meaning this right-hand side is not unique, so $\log(z)$ is not well-defined, and hence not a function.
To resolve this we need to choose a \keyword{branch} of $\log(z)$, by which we mean deciding in what range we let the argument live; so long as it's any open interval of length $2 \pi$ we are fine.
For instance, between $0$ and $2 \pi$ would be fine, as would $-\pi$ to $\pi$.

\topic{Analytic functions}

\begin{definition}[Complex differentiable]
	Let $G \subset \C$ be an open set.
	A function $f \colon G \to \C$ is \keyword{differentiable}\index{complex differentiability} at $z \in G$ if
	\[
		f'(z) \coloneqq \lim_{h \to 0} \frac{f(z + h) - f(z)}{h}
	\]
	exists.
\end{definition}
Note that, crucially, since $h$ is a complex number, this limit must exist along \emph{all} paths.

\begin{definition}[Analytic function]
	Let $G \subset \C$ be an open set.
	A function $f \colon G \to \C$ is \keyword{analytic}\index{analytic function} if it is continuously differentiable in $G$, i.e., $f'(z)$ exists and is continuous for all $z \in G$.
\end{definition}

\begin{remark}\label{remark:analytic}
	\begin{items}
		\item\label{remark:analytic:1} Every (complex) differentiable function is analytic (so the requirement of \emph{continuously} above is not necessary).
		This, of course, is \emph{not} true in real analysis.
		Take, for example,
		\[
			f(x) = \begin{cases}
				x^2 \sin(\frac{1}{x}), & \text{if } x \neq 0, \\
				0, & \text{if } x = 0.
			\end{cases}
		\]
		Then $f'(x)$ exists everywhere, but is not continuous at $x = 0$.

		\item Every analytic function is infinitely differentiable, and has a power series representation.
		This, again, is \emph{not} true in the real case.
		Take, say,
		\[
			f(x) = \begin{cases}
				0, & \text{if } x \leq 0, \\
				e^{-\frac{1}{x^2}}, & \text{if } x > 0.
			\end{cases}
		\]
		Then $f^{(n)}(0) = 0$ for all $n \in \N$, meaning that the Taylor series centred at $x = 0$ is identically zero, but $f$ is not identically zero near $0$, so $f(x)$ has no power series there (or, another way of putting it, the radius of convergence of its power series around $x = 0$ is zero).
	\end{items}
\end{remark}

\begin{exercise}
	Show that $f(z) = \abs{z}^2$ has a derivative only at $z = 0$.
\end{exercise}

\begin{proposition}
	Suppose that $\sum\limits_{n = 0}^\infty a_n (z - a)^n$ has radius of convergence $R > 0$.
	Then $f(z) \coloneqq \sum\limits_{n = 0}^\infty a_n (z - a)^n$ is analytic in $\abs{z - a} < R$, and moreover
	\[
		f^{(k)}(z) = \sum_{n = 0}^\infty n (n - 1) \dots (n - k + 1) a_n (z - a)^{n - k}
	\]
	for $\abs{z - a} < R$ (meaning that the derivative can be taken termwise), and in addition
	\[
		a_n = \frac{f^{(n)}(a)}{n!}.
	\]
\end{proposition}

\begin{example}
	As expected we have
	\[
		e^z = \sum_{n = 0}^\infty \frac{z^n}{n!}
	\]
	for all $z \in \C$, i.e., $R = \infty$.

	Also as expected,
	\[
		(e^z)' = \sum_{n = 1}^\infty \frac{n z^n}{n!} = \sum_{n = 0}^\infty \frac{z^n}{n!} = e^z
	\]
	by reindexing the first sum.

	Hence this is analytic.
\end{example}

\begin{example}
	Similarly, we then get
	\[
		\sin(z) = \frac{e^{i z} - e^{- i z}}{2 i} = z - \frac{z^3}{3!} + \frac{z^5}{5!} - \dots + (-1)^n \frac{z^{2 n - 1}}{(2 n - 1)!} + \dots
	\]
	and
	\[
		\cos(z) = \frac{e^{i z} + e^{- i z}}{2} = 1 - \frac{z^2}{2!} + \frac{z^4}{4!} - \dots + (-1)^n \frac{z^{2 n}}{(2 n)!} + \dots ,
	\]
	again with infinite radius of convergence.
	Differentiating termwise we thus, once again as expected, get $(\sin(z))' = \cos(z)$ and $(\cos(z))' = - \sin(z)$, making the two analytic.
\end{example}

\begin{example}
	Take a branch of $f(z) = \log(z)$, e.g., $-\pi < \arg(z) < \pi$.
	Then $f(z)$ is analytic in this branch, and $f'(z) = \frac{1}{z}$.
\end{example}

\topic{Cauchy--Riemann equations}

Suppose $f \colon G \to \C$ is analytic, and write $f$ as a function of the real and imaginary parts of $z$, i.e., $f(z) = f(x + i y) = f(x, y) = u(x, y) + i v(x, y)$, where then naturally $u, v \colon G \to \R$.

Since $f$ is analytic, the derivative
\[
	f'(z) = \lim_{h \to 0} \frac{f(z + h) - f(z)}{h}
\]
exists, and so in particular exists along any particular given path.
We will compute it along two paths, namely parallel to the real axis and parallel to the imaginary axis.
In other words, for $h = \delta \in \R$ and for $h = i \delta$, $\delta \in \R$.

First, then,
\begin{align*}
	f'(z) &= \lim_{\delta \to 0} \frac{f(z + \delta) - f(z)}{\delta} \\
	&= \lim_{\delta \to 0} \frac{(u(x + \delta, y) - u(x, y)) + i (v(x + \delta, y) - v(x, y))}{\delta} \\
	&= \frac{\partial u}{\partial x} (x, y) + i \frac{\partial v}{\partial x} (x, y).
\end{align*}
Similarly, along the imaginary direction,
\begin{align*}
	f'(z) &= \lim_{\delta \to 0} \frac{(u(x, y + \delta) - u(x, y)) + i (v(x, y + \delta) - v(x, y))}{i \delta} \\
	&= \frac{\partial v}{\partial y} (x, y) - i \frac{\partial u}{\partial y} (x, y).
\end{align*}
Since those are equal, we compare real and imaginary parts and see that
\[
	\begin{cases}
		\frac{\partial u}{\partial x} = \frac{\partial v}{\partial y} \\
		\frac{\partial u}{\partial y} = - \frac{\partial v}{\partial x},
	\end{cases}
\]
known as the \keyword{Cauchy--Riemann equations}\index{Cauchy--Riemann equations}, which the real and imaginary parts of an analytic function must therefore satisfy.

Differentiating these relations again, we get
\[
	\frac{\partial^2 u}{\partial x^2} = \frac{\partial^2 v}{\partial x \partial y}
\]
and
\[
	\frac{\partial^2 u}{\partial y^2} = - \frac{\partial^2 v}{\partial y \partial x}.
\]
Taking for granted at the moment that the imaginary parts have continuous partials, so that the second partials in the right-hand sides are equal, we then get
\[
	\frac{\partial^2 u}{\partial x^2} + \frac{\partial^2 u}{\partial y^2} = 0,
\]
i.e., letting $\Delta = \frac{\partial^2}{\partial x^2} + \frac{\partial^2}{\partial y^2}$, the \keyword{Laplace operator}\index{Laplace operator}, we have $\Delta u = 0$, meaning that $u$ is \keyword{harmonic}\index{harmonic!function}.
Taking the opposite partials we can draw the same conclusion about $v$.

\begin{theorem}
	Let $u$ and $v$ be real-valued functions defines on an open set $G \subset \C$.
	Suppose $u$ and $v$ have continuous partial derivatives.
	Then $f(z) \coloneqq u(z) + i v(z)$ is analytic if and only if $u$ and $v$ satisfy the Cauchy--Riemann equations.
\end{theorem}

\begin{remark}
	The function $v$ above is called a \keyword{harmonic conjugate}\index{harmonic!conjugate} of $u$.
\end{remark}

\begin{proof}[Proof sketch]
	Use the Cauchy--Riemann equations to show that the derivative of $f(z) = u(z) + i v(z)$ exists.
	The converse direction we proved by the previous discussion.
\end{proof}

\begin{theorem}
	Let $G \subset \C$ be a simply connected open set.
	If $u \colon G \to \C$ is a harmonic function, then $u$ has a harmonic conjugate, i.e., there exists a $v \colon G \to \C$ such that $f = u + i v$ is analytic.
\end{theorem}

\begin{proof}[Proof sketch]
	Assume $G$ is a ball, say $G = B(0, R)$.
	We need to find $v$ such that $u$ and $v$ satisfy the Cauchy--Riemann equations.
	In other words, we first need $\frac{\partial v}{\partial y} = \frac{\partial u}{\partial x}$.
	Defining
	\[
		v(x, y) = \int_0^y \frac{\partial u}{\partial x}(x, t) \, d t + \varphi(x)
	\]
	does the job, since $u$ is given.
	Now using the second equation, $\frac{\partial v}{\partial x} = - \frac{\partial u}{\partial y}$, we can solve for $\varphi(x)$, namely
	\[
		\frac{\partial v}{\partial x} = \int_0^y \frac{\partial^2 u}{\partial x^2}(x, t) \, d t + \varphi'(x).
	\]
	Since $u$ is harmonic, we can rewrite this as
	\[
		\int_0^y - \frac{\partial^2 u}{\partial y^2} (x, t) \, d t + \varphi'(x) = - \frac{\partial u}{\partial y} (x, y) + \frac{\partial u}{\partial y}(x, 0) + \varphi'(x).
	\]
	On the other hand, by the second Cauchy--Riemann equation, this equals
	\[
		- \frac{\partial u}{\partial y}(x, y),
	\]
	whence
	\[
		\varphi'(x) = \frac{\partial u}{\partial y}(x, 0),
	\]
	implying
	\[
		v(x, y) = \int_0^y \frac{\partial u}{\partial x}(x, t) \, d t + \int_0^x \frac{\partial u}{\partial y}(s, 0) \, d s.
	\]
	Of course if $G$ isn't a ball we might not be able to integrate along quite this path, but similar arguments work.
\end{proof}

\begin{exercise}\label{hw1.3}
	Let $G$ be an open subset of $\C$.
	Define $\conjugate{G} = \Set{z \given \conjugate{z} \in G}$.
	Suppose that $f \colon G \to \C$ is analytic.
	Show that $f^\star \colon \conjugate{G} \to \C$ defined by $f^\star(z) = \conjugate{f(\conjugate{z})}$ is also analytic.
\end{exercise}
