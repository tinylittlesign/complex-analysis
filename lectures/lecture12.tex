%!TEX root = ../lectures.tex

\lecture[September 26th, 2019]{Phragmén--Lindelöf Principle}

\topic{Further generalising the maximum modulus principle}

First let us establish a variant of the \nameref{thm6.6}:

\index{Hadamard three-circle theorem|(}
\begin{theorem}[Hadamard three-circle theorem]\label{thm6.9}
	Let $0 < R_1 < R_2 < \infty$.
	Suppose $f$ is analytic and not identically zero on the annulus $A = \Set{z \in \C \given R_1 < \abs{z} < R_2}$.
	For $R_1 < r < R_2$, define
	\[
		M(r) = \max_{0 \leq \theta \leq 2 \pi} \abs{f(r e^{i \theta})}.
	\]
	Then for $R_1 < r_1 < r < r_2 < R_2$, we have
	\[
		\log M(r) \leq \frac{\log r_2 - \log r}{\log r_2 - \log r_1} \log M(r_1) + \frac{\log r - \log r_1}{\log r_2 - \log r_1} \log M(r_2),
	\]
	i.e., $\log M(r)$ is a convex function of $\log r$.
\end{theorem}
\index{Hadamard three-circle theorem|)}

\begin{marginfigure}
	\centering
	\mfincludegraphics[width=\textwidth]{figures/l12fig69a.tikz}

	becomes

	\mfincludegraphics[width=\textwidth]{figures/l12fig69b.tikz}
	\caption{\label{thm69:fig} Transforming a vertical strip with $\exp$.}
\end{marginfigure}

\begin{proof}
	Noticing how the exponential function $\exp$ maps vertical lines at real part $\log r$ to circles centred at the origin with radius $r$, as illustrated in \autoref{thm69:fig}.

	This result now follows immediately from the \nameref{thm6.6}, since if we consider $g = f \circ \exp$, then $g(\log r + i \theta) = f(r e^{i \theta})$.
	Hence
	\[
		M(r) = \max_{0 \leq \theta \leq 2 \pi} \abs{f(r e^{i \theta})} = \sup_{-\infty \leq \theta \leq \infty} \abs{g(\log r + i \theta)},
	\]
	the logarithm of which is convex in $\log r$.
\end{proof}

\begin{exercise}
	Let $f$ be analytic in an annulus $R_1 < \abs{z} < R_2$ and not identically zero.
	Let
	\[
		I_2(r) = \frac{1}{2 \pi} \int_0^{2 \pi} \abs{f(r e^{i \theta})}^2 \, d \theta
	\]
	Show that $\log I_2(r)$ is a convex function of $\log r$ for $R_1 < r < R_2$.

	Moreover, if $f$ is a non-constant analytic function on $B(0; R)$, then $I_2(r)$ is strictly increasing.
\end{exercise}

\index{Hardy's theorem|(}
\begin{exercise}[(A special case of) Hardy's theorem]
	Let $f$ be a non-constant analytic function on $B(0; R)$.
	Show that
	\[
		I(r) = \frac{1}{2 \pi} \int_0^{2 \pi} \abs{f(r e^{i \theta})} \, d \theta
	\]
	is strictly increasing and $\log I(r)$ is a convex function of $\log r$.
\end{exercise}
\index{Hardy's theorem|)}

Notice how \autoref{cor6.8} of the \nameref{thm6.6} says that if a function $f$ is bounded by $M$ on vertical lines, then it is also bounded by $M$ between the vertical lines.

In general, to use the \nameref{thm6.1} to draw such a conclusion, we would require boundedness also at infinity, since the vertical strip is unbounded.
The following is a powerful generalisation of this:

\index{Phragmén--Lindelöf principle|(}
\begin{theorem}[Phragmén--Lindelöf principle]\label{thm6.10}
	Let $G$ be a simply connected region.
	Let $f \colon G \to \C$ be an analytic function, and $\varphi \colon G \to \C$ be analytic and bounded in $G$ and $\varphi(z) \neq 0$ for all $z \in G$.

	Suppose $\partial_\infty G = A \cup B$ such that
	\begin{items}
		\item $\displaystyle \limsup_{z \to a} \abs{f(z)} \leq M$ for all $a \in A$, and
		\item $\displaystyle \limsup_{z \to b} \abs{\varphi(z)}^\eta \abs{f(z)} \leq M$ for all $b \in B$ and all $\eta > 0$ (in other words, $f$ isn't necessarily bounded on $B$, but its growth is controlled by $\varphi$).
	\end{items}
	Then $\abs{f(z)} \leq M$ for all $z$ in $G$.
\end{theorem}
\index{Phragmén--Lindelöf principle|)}

\begin{remark}
	In applications, one often takes $A = \partial G$ and $B = \Set{\infty}$, so we only need to control growth at infinity, and on the usual boundary we are just bounded.
\end{remark}

\begin{proof}
	Let $\abs{\varphi(z)} \leq K$ for all $z \in G$.
	Since $G$ is simply connected and $\varphi(z) \neq 0$ for all $z \in G$, we can take $\varphi(z) = e^{h(z)}$ for some analytic function $h \colon G \to \C$.
	Now define $g(z) = e^{\eta h(z)}$ so that $\abs{g(z)} = \abs{\varphi(z)}^\eta$.

	Consider the function $F \colon G \to \C$ defined by
	\[
		F(z) = \frac{f(z) g(z)}{K^\eta},
	\]
	which is analytic on $G$.
	We wish to use the \nameref{thm6.1}, and so we should check that $F(z)$ is bounded on $\partial_\infty G$, which by construction is $A \cup B$.
	To do this, first notice how
	\[
		\abs{F(z)} = \frac{\abs{f(z)} \abs{g(z)}}{K^\eta} \leq \frac{\abs{f(z)} K^\eta}{K^\eta} = \abs{f(z)},
	\]
	so for $a \in A$,
	\[
		\limsup_{z \to a} \abs{F(z)} \leq \limsup_{z \to a} \abs{f(z)} \leq M.
	\]
	Similarly, for $b \in B$,
	\[
		\limsup_{z \to b} \abs{F(z)} = \limsup_{z \to b} \frac{\abs{f(z)} \abs{g(z)}}{K^\eta} \leq \frac{1}{K^\eta} M
	\]
	by choice of $g$.
	Hence by the \nameref{thm6.1},
	\[
		\abs{F(z)} \leq \max\Set[\Big]{M, \frac{M}{k^\eta}}
	\]
	for all $z \in G$, and therefore
	\[
		\abs{f(z)} \leq \frac{K^\eta}{\abs{\varphi(z)}^\eta} \max\Set[\Big]{M, \frac{M}{K^\eta}}
	\]
	for all $z \in G$ since $\varphi(z) \neq 0$.
	By taking $\eta \to 0$, this bound goes to $M$, and so $\abs{f(z)} \leq M$ for all $z \in G$.
\end{proof}

This gives us a general tool for bounding functions on unbounded domains, more powerful than the \nameref{thm6.1} since we no longer require boundedness at infinity, only some sufficiently good growth conditions.

For example, both as an example of how to use the theorem and the common practice of taking $B = \Set{\infty}$ as mentioned in the remark:

\begin{corollary}\label{cor6.11}
	Let $G$ be the sector $G = \Set{z \given \abs{\arg z} < \frac{\pi}{2 a}}$ for $a \geq \frac{1}{2}$.
	Suppose $f$ is analytic in $G$ and there exists some $M > 0$ such that
	\[
		\limsup_{z \to w} \abs{f(z)} \leq M
	\]
	for all $w \in \partial G$.
	Suppose moreover there exists some $p > 0$ and $0 \leq b < a$ such that
	\[
		\abs{f(z)} \leq p \exp(\abs{z}^b)
	\]
	for all $z \in G$ with $\abs{z}$ sufficiently large.
	Then $\abs{f(z)} \leq M$ for all $z \in G$.
\end{corollary}

\begin{marginfigure}
	\mfincludegraphics[width=\textwidth]{figures/l12fig611a.tikz}
	\caption{\label{cor611:fig} Sector $G$ of angle $\frac{\pi}{a}$.}
\end{marginfigure}

\begin{remark}\label{remark:cor6.11}
	The corollary remains true if we replace $G$ by any sector $S$ of angle $\frac{\pi}{a}$.
	To see this, consider
	\[
		\begin{tikzcd}
			g \colon G \arrow[r, "e^{i\theta}"] & S \arrow[r, "f"] & \C,
		\end{tikzcd}
	\]
	i.e., $g(z) = f(e^{i \theta} z)$.
	Then $\sup \abs{g(z)} = \sup \abs{f(z)}$, and for a given $f$, we can choose $\theta$ such that $g$ is in the original sector $G$ of the same angle, and apply the corollary there.
\end{remark}

\begin{proof}
	Let $b < c < a$, and define $\varphi(z) = \exp(-z^c)$ for $z \in G$.
	Then $\varphi(z) \neq 0$ since it is an exponential function, and writing $z = r e^{i \theta} \in G$ with $\abs{\theta} < \frac{\pi}{2 a}$ we have
	\[
		\Re(z^c) = \Re(r^c e^{i c \theta}) = r^c \cos(c \theta).
	\]
	Since $0 < c < a$ and $\abs{\theta} < \frac{\pi}{2 a}$, we have $\abs{c \theta} < \frac{\pi}{2} \cdot \frac{c}{a} < \frac{\pi}{2}$, whence $\cos(c \theta) \geq \rho > 0$.
	This means that
	\[
		\Re(- z^c) \leq -r^c \rho < 0,
	\]
	which further gives us
	\[
		\abs{\varphi(z)} = \exp(\Re(-z^c)) \leq 1.
	\]
	Hence for any $\eta > 0$, $z = r e^{i \theta} \in G$, we have
	\[
		\abs{f(z)} \abs{\varphi(z)}^\eta \leq p \exp(\abs{z}^b) \exp(-r^c \rho \eta) = p \exp(r^b - (\rho \eta) r^c).
	\]
	Since $c > b$ by choice, the second term in the exponential is dominant, so as $r \to \infty$ the inside goes to $-\infty$ and so the exponential as a whole goes to $0$.
	Thus
	\[
		\limsup_{\abs{z} \to \infty} \abs{f(z)} \abs{\varphi(z)}^\eta = 0,
	\]
	and hence by the \nameref{thm6.10},
	\[
		\abs{f(z)} \leq \max\Set{M, 0} = M
	\]
	for all $z \in G$.
\end{proof}

\begin{remark}
	In this corollary, for a sector of angle $\frac{\pi}{a}$, we assume $\abs{f(z)}$ is bounded by $\exp(\abs{z}^b)$, specifically with $b < a$.
	If we take $b = a$, the corollary is still true, but the estimate needed is more subtle.

	Looking at it, the key reason in our above proof where we use $b < a$ is to pick $b < c < a$, which in the end makes $r^c$ the dominant term in $\exp(r^b - (\rho \eta) r^c)$.
	If $b = a\, (= c)$, by this argument we get $\exp((1 - \rho \eta) r^a)$, so for small $\eta$, this goes to infinity as $r \to \infty$.
\end{remark}

\begin{exercise}
	Let $G = \Set{z \given \Re(z) > 0}$ and let $f \colon G \to \C$ be analytic such that $f(1) = 0$.
	Suppose that
	\[
		\limsup_{z \to w} \abs{f(z)} \leq M
	\]
	for $w \in \partial G$, and suppose that for every $\delta > 0$ with $0 < \delta < 1$, there is a constant $P$ such that
	\[
		\abs{f(z)} \leq P \exp(\abs{z}^{1 - \delta})
	\]
	for all $z \in G$.
	Prove that
	\[
		\abs{f(z)} \leq M \left ( \frac{(1 - x)^2 + y^2}{(1 + x)^2 + y^2} \right )^{1/2},
	\]
	for all $z = x + i y \in G$.
\end{exercise}

\begin{corollary}\label{cor6.12}
	Let $G = \Set{z \given \abs{\arg z} < \frac{\pi}{2 a}}$ for $a \geq \frac{1}{2}$.
	Suppose
	\[
		\limsup_{z \to w} \abs{f(z)} \leq M
	\]
	for all $w \in \partial G$.
	Suppose for any $\delta > 0$ there exists a constant $p_\delta > 0$ such that
	\[
		\abs{f(z)} \leq p_\delta \exp(\delta \abs{z}^a)
	\]
	for all $z \in G$ with $\abs{z}$ sufficiently large.
	Then $\abs{f(z)} \leq M$ for all $z \in G$.
\end{corollary}

\begin{proof}
	For any $\varepsilon > 0$, consider $F \colon G \to \C$ defined by
	\[
		F(z) = f(z) \exp(-\varepsilon z^a).
	\]
	Choose $0 < \delta < \varepsilon$ and $p_\delta > 0$ so that $\abs{f(z)} \leq p_\delta \exp(\delta \abs{z}^a)$.
	Then for any $x > 0$, $x \in \R$, we have
	\[
		\abs{F(x)} \leq p_\delta \exp((\delta - \varepsilon) x^a) \to 0
	\]
	as $x \to \infty$ since $\delta - \varepsilon < 0$.
	The idea here is this: the original approach requires $\abs{f(z)} \leq p \exp(\abs{z}^b)$ with $b < a$, where the sector is of angle $\frac{\pi}{a}$, but we currently have $b = a$.
	What we have just shown is that $F(z)$ is bounded on the positive real axis, so we can split our sector of angle $\frac{\pi}{a}$ into two sectors both of angle $\frac{\pi}{2 a}$---see \autoref{cor612:fig}---where now $b = a < 2 a$ permits us to use the previous result.

	In other words, let $H_+ = \Set{z \given 0 < \arg z < \frac{\pi}{2 a}}$ and $H_- = \Set{ z \given -\frac{\pi}{2 a} < \arg z < 0}$, and let
	\[
		M_1 \coloneqq \sup_{0 < x < \infty} \abs{F(x)} < \infty
	\]
	since by the above discussion $F(x)$ is bounded at infinity, and define $M_2 = \max\Set{M_1, M}$.

	\begin{marginfigure}
		\mfincludegraphics[width=\textwidth]{figures/l12fig612b.tikz}
		\caption{\label{cor612:fig} Splitting sector $G$ into halves of angle $\frac{\pi}{2 a}$.}
	\end{marginfigure}

	Then for $w \in \partial H_+$ or $w \in \partial H_-$ we have
	\[
		\limsup_{z \to w} \abs{F(z)} \leq M_2
	\]
	since it is bounded by $M$ on the original boundary and by $M_1$ on the real axis.
	Hence by \autoref{cor6.11} and \autoref{remark:cor6.11} about how it works for any sector, $\abs{F(z)} \leq M_2$ for all $z \in H_+$ and $z \in H_-$, and hence $\abs{F(z)} \leq M_2$ for all $z \in G$.

	Finally, we claim that $M_2 = M$, i.e., $M \geq M_1$.
	To see this, suppose $M < M_1$.
	Then there exists some $x \in \R$ such that $\abs{F(x)} \geq \abs{F(z)}$ for all $z \in G$, in other words $F(z)$ attains its maximum in the interior of $G$.
	The \nameref{thm6.1} then implies $F(z)$ is constant, which is a contradiction.

	Hence $\abs{F(z)} \leq M$ for all $z \in G$, and therefore $\abs{f(z)} \abs{\exp(-\varepsilon z^a)} \leq M$, implying in the end
	\[
		\abs{f(z)} \leq M \exp(\varepsilon z^a)
	\]
	which as $\varepsilon \to 0$ goes to $M$.
\end{proof}
