%!TEX root = ../lectures.tex

\lecture[November 19th, 2019]{The Great Picard Theorem}

\topic{The range of entire functions, revisited}

The \nameref{thm9.14} will follow almost immediately from the following slightly technical theorem:

% Cheat because I'm lazy. Remove at some point.
%TODO.
\clearpage

\index{Montel--Carathéodory theorem|(}
\begin{theorem}[Montel--Carathéodory theorem]\label{thm9.13}
	Let $G$ be a region.
	Let
	\[
		\mathcal{F} = \Set{f \colon G \to \C\ \text{analytic} \given \text{$f(z)$ omits $0$ and $1$}}.
	\]
	Then $\mathcal{F}$ is normal in $C(G, \C_\infty)$.
\end{theorem}
\index{Montel--Carathéodory theorem|)}

\begin{proof}
	Fix $z_0 \in G$ and let $\mathcal{G} = \Set{f \in \mathcal{F} \given \abs{f(z)} \leq 1}$ and $\mathcal{H} = \Set{f \in \mathcal{F} \given \abs{f(z)} > 1}$.
	Then $\mathcal{F} = \mathcal{G} \cup \mathcal{H}$ and if we can show that $\mathcal{G}$ and $\mathcal{H}$ are normal in $C(G, \C_\infty)$, then so is $\mathcal{F}$.

	We will show in particular that $\mathcal{G}$ is normal in $H(G)$, and $\mathcal{H}$ is normal in $C(G, \C_\infty)$, one at a time.

	First, to show that $\mathcal{G}$ is normal in $H(G)$, recall how by \nameref{thm7.10} it suffices to show that $\mathcal{G}$ is locally bounded.

	To this end, let $a \in G$ and let $\gamma$ be a path from $z_0$ to $a$.
	Take disks $\closure{D_0}, \closure{D_1}, \dots, \closure{D_n}$ in $G$, centred at $z_0, z_1, \dots, z_n = a \in \Set{\gamma}$, so that $z_{k - 1}, z_k \in D_{k - 1} \cup D_k$ for all $1 \leq k \leq n$.
	This is possible by taking $z_{k - 1}$ and $z_k$ sufficiently close to one another, and since $\gamma$ is compact, we can find a finite subset of them that does the job.

	Now by applying a shifted version of \nameref{thm9.11} on $D_0$, we see that there exists some constant $C_0$ such that $\abs{f(z)} \leq C_0$ for all $z \in D_0$ and all $f \in \mathcal{G}$.

	In particular, since by construction $z_1 \in D_0$, this gives us $\abs{f(z_1)} \leq C_0$ for all $f \in \mathcal{G}$.
	This lets us apply \nameref{thm9.11} again to $D_1$, whence $\abs{f(z)} \leq C_1$ for all $f \in \mathcal{G}$ and all $z \in D_1$ for some constant $C_1$, and in particular $\abs{f(z_2)} \leq C_1$ for all $f \in \mathcal{G}$.

	Repeat this argument on $D_2$, $D_3$, and so on, until we get for $D_n$ that $\mathcal{G}$ is uniformly bounded on $D_n = B(a, r)$ for some radius $r$.
	Hence $\mathcal{G}$ is locally bounded, since $a \in G$ is arbitrary, and therefore $\mathcal{G}$ is normal.

	This leaves showing that $\mathcal{H}$ is normal in $C(G, \C_\infty)$.
	If $f \in \mathcal{H}$ is analytic, then $\frac{1}{f}$ is analytic on $G$ (since $f$ omits zero, being bounded below by $1$).
	Moreover, $\frac{1}{f(z)} \neq 0$ since $f$ is analytic, hence having no poles, and $\frac{1}{f(z)} \neq 1$ since $f(z) \neq 1$.
	Finally, $\abs{\frac{1}{f(z)}} < 1$ since $\abs{f(z) > 1}$.

	All by way of saying: $\tilde{\mathcal{H}} \coloneqq \Set{\frac{1}{f} \given f \in \mathcal{H}} \subset \mathcal{G}$.
	Since $\mathcal{G}$ is normal, $\tilde{\mathcal{H}}$ must be normal too.

	In other words, if $\Set{f_n} \subset \mathcal{H}$, then there exists a subsequence $\Set{f_{n_k}}$ such that $\frac{1}{f_{n_k}} \to h \in H(G)$.

	By \autoref{cor7.8} of \nameref{thm7.7}, this means either $h = 0$ identically or $h(z) \neq 0$ for all $z \in G$.
	In the first case we get $f_{n_k} \to \infty$ in $C(G, \C_\infty)$, and in the second case we see that $\frac{1}{h}$ is analytic on $G$, and $f_{n_k} \to \frac{1}{h}$ in $H(G)$, so in particular in $C(G, \C_\infty)$.
	In either case we have a subsequence of $\Set{f_n} \subset \mathcal{H}$ converging in $C(G, \C_\infty)$, so $\mathcal{H}$ is normal.
\end{proof}

With this in hand we are equipped to prove

\index{Great Picard theorem|(}\index{Picard's theorem!great|see {Great Picard theorem}}
\begin{theorem}[Great Picard theorem]\label{thm9.14}
	Let $G$ be a region and $a \in G$.
	Let $f$ be analytic on $G \setminus \Set{a}$ and suppose $g$ has an essential singularity at $z = a$.
	Then in each neighbourhood of $z = a$, $f(z)$ assumes each complex number, with one possible exception, infinitely many times.
\end{theorem}
\index{Great Picard theorem|)}

\begin{remark}
	This improves the \nameref{thm5.4}, which says that the image of each neighbourhood of an essential singularity is dense in $\C$.
\end{remark}

\begin{proof}
	Without loss of generality, assume $f$ has an essential singularity at $z = 0$ (else shift it).
	Suppose there exists some $R > 0$ such that $f(z)$ omits two values on $0 < \abs{z} < R$.
	Again we can assume those two values are $0$ and $1$, i.e., $f(z) \neq 0$ and $f(z) \neq 1$ for all $0 < \abs{z} < R$, else normalise as in the proof of the \nameref{thm9.10}.

	Let $G = B(0, R) \setminus \Set{0}$ and define $f_n \colon G \to \C$ by $f_n(z) = f(\frac{z}{n})$.
	Then $f_n$ is analytic on $G$ since $f$ is and $f_n(z) \neq 0$ for all $z \in G$ since $f(z) \neq 0$ on $G$.

	By \nameref{thm9.13}, $\set{f_n}$ is normal in $C(G, \C_\infty)$, meaning that there exists a subsequence $\Set{f_{n_k}}$ such that $f_{n_k} \to \varphi$ in $C(G, \C_\infty)$.
	In particular, $f_{n_k} \to \varphi$ uniformly on any compact subset of $G$, so in particular on $\abs{z} = \frac{1}{2} R$.

	Note how, since $\Set{f_{n_k}} \subset H(G)$, we have again by \autoref{cor7.8} of \nameref{thm7.7} that either $\varphi$ is analytic on $G$ or $\varphi = \infty$ identically.

	In the former case, let
	\[
		M = \max_{\abs{z} = \frac{1}{2} R} \abs{\varphi(z)},
	\]
	which exists since $\varphi$ is analytic and $\abs{z} = \frac{1}{2} R$ is compact.
	Then for $\abs{z} = \frac{1}{2} R$,
	\[
		\abs[\Big]{f\Bigl( \frac{z}{n_k})} = \abs{f_{n_k}(z)} \leq \abs{f_{n_k}(z) - \varphi(z)} + \abs{\varphi(z)}.
	\]
	The first term in the right-hand side goes to $0$ uniformly, and the second term is bounded uniformly by $M$, so for $n_k$ sufficiently large we have $\abs{f(\frac{z}{n_k})} \leq 2 M$.

	Hence $\abs{f(z)} \leq 2 M$ for $\abs{z} = \frac{R}{2 n_k}$ for $n_k$ large, so $f(z)$ is uniformly bounded on $B(0, r) \setminus \Set{0}$ for some $0 < r < R$.
	This means $f(z)$ has a removable singularity at $z = 0$, which is a contradiction.

	Similarly, in the latter case, assume $\varphi = \infty$.
	In this case, $f_{n_k} \to \infty$ uniformly on $\abs{z} = \frac{1}{2} R$, which means $f(\frac{z}{n_k}) \to \infty$ uniformly on $\abs{z} = \frac{1}{2} R$.
	In other words,
	\[
		\lim_{z \to \infty} \abs{f(z)} = \infty,
	\]
	meaning that $f$ has a pole at $z = 0$, which is again a contradiction.

	Hence $f$ cannot omit two values in $\C$, meaning it can omit \emph{at most} one value.

	For the second part of the theorem we need to show that, apart from this possible exceptional point, all points are attained infinitely many times.

	Suppose, therefore, that two values are assumed by $f$ only finitely many times.
	This means that there are some finite set of preimages of those two points in $G$, which in turn means we must be able to find some sufficiently small punctured disk $B(0, r) \setminus \Set{0}$, $0 < r < R$, not containing any of those preimages.
	Hence on this smaller punctured disk, $f$ omits those two values, which by the previous discussion is impossible.
\end{proof}

An immediate corollary of this is

\begin{corollary}\label{cor9.15}
	If $f$ has an isolated singularity at $z = 0$ and if there are two values that are not assumed by $f(z)$ infinitely many times, then $z = a$ is either a pole or a removable singularity (i.e., $z = a$ cannot be an essential singularity).
\end{corollary}

An almost as immediate corollary, and the result we promised a bit ago, is this:

\begin{corollary}\label{cor9.16}
	If $f$ is an entire function that is not a polynomial, then $f(z)$ assumes every complex number, with one possible exception, infinitely many times.
\end{corollary}

\begin{exercise}
	Suppose $f$ is a one-to-one entire function.
	Show that $f(z) = a z + b$ for some $a, b \in \C$ with $a \neq 0$.
\end{exercise}

\begin{proof}
	Since $f$ is entire and not a polynomial, it has a power series expansion about $z = 0$ that never ends.
	Consequently, the Laurent expansion of $g(z) = f(\frac{1}{z})$ extends infinitely in the negative direction, so $g$ has an essential singularity at $z = 0$.

	Therefore by the \nameref{thm9.14} $g(z)$ assumes each complex number, with one possible exception, infinitely many times.
	But $f$ and $g$ have the same image, so $f(z)$ does too.
\end{proof}

\topic{Runge's theorem}

Recall how in real analysis, Weierstrass theorem\index{Weierstrass theorem} says that every continuous function $f$ on a compact set in $\R$ can be approximated uniformly by polynomials.
(One way to prove this is by way of harmonic analysis: show that they can be approximated by Fourier series, i.e., linear combinations of trigonometric functions, and then approximate those trigonometric functions by their Taylor polynomials).

A natural question to ask, then, is this: let $K \subset \C$ be compact and let $G$ be a neighbourhood of $K$.
Suppose $f \in H(G)$.
Can $f$ be approximated uniformly by polynomials on $K$?

The answer, in general, is no.
Consider the following two examples:

\begin{example}
	Let $G = B(0, R)$.
	For any $f \in H(G)$, we have a power series expansion
	\[
		f(z) = \sum_{n = 0}^\infty a_n z^n,
	\]
	which converges uniformly on any compact subset $K \subset G$.
	Taking the partial sums
	\[
		P_k(z) = \sum_{n = 0}^k a_n z_n,
	\]
	we then have $P_k \to f$ uniformly on $K$ as $k \to \infty$.
	Hence in this particular case, the answer to the above question is yes.
\end{example}

On the other hand,

\begin{example}
	Let $G = B(0, R) \setminus \Set{0}$, and let $f(z) = \frac{1}{z} \in H(G)$ (it is holomorphic since $G$ specifically excludes the pole).
	Let $K = \Set{z \given \abs{z} = \frac{1}{2} R} \subset G$, which is compact.

	For any polynomial $P(z)$,
	\[
		\int_K P(z) \, d z = 0
	\]
	since $K$ encloses no pole of $P(z)$ (it has no poles).
	On the other hand,
	\[
		\int_K f(z) \, d z = \int_K \frac{1}{z} \, d z = 2 \pi i
	\]
	since $K$ encloses the pole at $z = 0$ of $f$.

	Now if $P_n \to f$ uniformly on $K$, we must have
	\[
		\int_K P_n(z) \, d z \to \int_K f(z) \, d z
	\]
	but the left-hand side is constantly $0$, and the right-hand side if $2 \pi i \neq 0$, so $f$ cannot be approximated uniformly by polynomials on $K$.
\end{example}

What these examples show is that, in fact, whether $f$ can be approximated uniformly by polynomials or not is a topological property: it depends on whether $\C \setminus K$ is connected or not.

What \nameref{thm10.4}\index{Runge's theorem} says is this, which we will work toward proving:
\begin{items}
	\item If $f \in H(G)$, then $f$ can be approximated uniformly by \emph{rational functions} on $K$; and
	\item If $\C \setminus K$ is connected, then $f \in H(G)$ can be approximated uniformly by \emph{polynomials} on $K$.
\end{items}
