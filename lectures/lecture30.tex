%!TEX root = ../lectures.tex

\lecture[December 5th, 2019]{Perron's Formula}

\topic{Uniqueness of Dirichlet series}

\begin{remark}
	As a consequence of \autoref{prop12.2}, we infer:
	\begin{items}
		\item The region of convergence of a Dirichlet series is a half-plane.
		\item There exists a unique $\sigma_c \in \interval{-\infty}{\infty}$ such that $D(s)$ converges for $\Re(s) > \sigma_c$ and diverges for $\Re(s) < \sigma_c$.
		Such a $\sigma_c$ is called the \keyword{abscissa of convergence}\index{abscissa of convergence} of $D(s)$.
	\end{items}
\end{remark}

Notice how, since absolute convergence implies convergence, $\sigma_c \leq \sigma_{ac}$.
However they need not be equal.

\begin{example}
	Consider the alternating harmonic series
	\[
		D(s) = \sum_{n = 1}^\infty \frac{(-1)^n}{n^2}.
	\]
	This has abscissa of absolute convergence $\sigma_{ac} = 1$, but abscissa of convergence $\sigma_c = 0$ (from \keyword{Dirichlet's test}\index{Dirichlet's test}: $a_n = (-1)^n$ has bounded average, and $\frac{1}{n^s}$ is decreasing to $0$ in magnitude for $\Re(s) > 0$).
\end{example}

In this example $\sigma_{ac}$ and $\sigma_c$ differ by one.
This is the worst case scenario:

\begin{proposition}\label{prop12.3}
	Let
	\[
		D(s) = \sum_{n = 1}^\infty \frac{a_n}{n^s}
	\]
	with abscissa of absolute convergence $\sigma_{ac}$ and abscissa of convergence $\sigma_c$.
	Then $0 \leq \sigma_{ac} - \sigma_c \leq 1$.
\end{proposition}

\begin{proof}
	The left-hand side inequality is trivial since $\sigma_c \leq \sigma_{ac}$.
	Hence it remains to show $\sigma_{ac} \leq 1 + \sigma_c$, i.e., show that
	\[
		\sum_{n = 1}^\infty \frac{a_n}{n^{1 + \sigma_c + \varepsilon}}
	\]
	converges absolutely for all $\varepsilon > 0$.

	Since $\sigma_c$ is the abscissa of convergence,
	\[
		D\Bigl( \sigma_c + \frac{\varepsilon}{2} \Bigr) = \sum_{n = 1}^\infty \frac{a_n}{n^{\sigma_c + \frac{\varepsilon}{2}}}
	\]
	converges, meaning that the individual terms must go to zero.
	In other words,
	\[
		\lim_{n \to \infty} \frac{a_n}{n^{\sigma_c + \frac{\varepsilon}{2}}} = 0.
	\]
	Hence there exists some $N \in \N$ such that for $n > N$,
	\[
		\abs[\Big]{\frac{a_n}{n^{\sigma_c + \frac{\varepsilon}{2}}}} < 1.
	\]
	Consequently we consider the tail
	\[
		\sum_{n = N}^\infty \abs[\Big]{\frac{a_n}{n^{1 + \sigma_c + \varepsilon}}} = \sum_{n = N}^\infty \abs[\Big]{\frac{a_n}{n^{\sigma_c + \frac{\varepsilon}{2}}}} \cdot \frac{1}{n^{1 + \frac{\varepsilon}{2}}} \leq \sum_{n = N}^\infty \frac{1}{n^{1 + \frac{\varepsilon}{2}}} < \infty,
	\]
	so we have the absolute convergence we were looking for.
\end{proof}

\index{Dirichlet series!uniqueness theorem|(}
\begin{theorem}[Uniqueness theorem for Dirichlet series]\label{thm12.4}
	Let
	\[
		D(s) = \sum_{n = 1}^\infty \frac{a_n}{n^s} \qquad \text{and} \qquad F9s) = \sum_{n = 1}^\infty \frac{b_n}{n^s}
	\]
	with abscissae of absolute convergence less than $\sigma$.
	Suppose there exist $\Set{s_k} \subset \Set{s \given \Re(s) > \sigma}$ such that
	\begin{items}
		\item $\Re(s_k) = \sigma_k \to \infty$ as $k \to \infty$ and
		\item $D(s_k) = F(s_k)$ for all $k$.
	\end{items}
	Then $a_n = b_n$ for all $n \in \N$.
\end{theorem}
\index{Dirichlet series!uniqueness theorem|)}

\begin{proof}
	Define a new Dirichlet series
	\[
		G(s) = D(s) - F(s) = \sum_{n = 1}^\infty \frac{a_n - b_n}{n^s} = \sum_{n = 1}^\infty \frac{c_n}{n^s},
	\]
	i.e., let $c_n = a_n - b_n$.
	We wish to show that $c_n = 0$ for all $n \in N$.

	Notice how, by assumption, $G(s_k) = 0$ for all $k$.
	Now suppose $c_n \neq 0$ for some $n$.
	Then there must be a smallest index for which $c_n$ is nonzero, so let $N$ be the smallest integer such that $c_N \neq 0$, meaning that $c_n = 0$ for all $n < N$.
	Then
	\[
		G(s) = \sum_{n = N}^\infty \frac{c_n}{n^s} = \frac{c_N}{N^s} + \sum_{n = N + 1}^\infty \frac{c_n}{n^s}.
	\]
	Since $G(s_k) = 0$, we must have
	\[
		\frac{c_N}{N^{s_k}} + \sum_{n = N + 1}^\infty \frac{c_n}{n^{s_k}} = 0,
	\]
	which, if we rearrange and take absolute values, becomes
	\[
		\abs{c_N} \leq N^{\sigma_k} \sum_{n = N + 1}^\infty \frac{\abs{c_n}}{n^{\sigma_k}} \leq \frac{N^{\sigma_k}}{(N + 1)^{\sigma_k - \sigma - \varepsilon}} \sum_{n = N + 1}^\infty \frac{abs{c_n}}{n^{\sigma + \varepsilon}}.
	\]
	The factor in front of the summation goes to $0$ as $\sigma_k \to \infty$, and the sum is finite since we are in the region of absolute convergence.
	Hence all of this goes to $0$, implying that $c_N = 0$, which is a contradiction.
\end{proof}

\topic{Perron's formula}

\begin{lemma}\label{lem12.5}
	Let $c > 0$ and $y > 0$.
	Then we have
	\[
		\frac{1}{2 \pi i} \int_{(c)} y^w \frac{d w}{w} = \begin{cases}
			1, & \text{if $y > 1$} \\
			0, & \text{if $0 < y < 1$}.
		\end{cases}
	\]
\end{lemma}

\begin{remark}
	By integrating over $(c)$ we mean to integrate along the vertical line at $\Re(s) = c$, i.e., in this example,
	\[
		\frac{1}{2 \pi i} \int_{(c)} y^w \frac{d w}{w} = \frac{1}{2 \pi i} \int_{c - i \infty}^{c + i \infty} y^w \frac{d w}{w}.
	\]
\end{remark}

\begin{marginfigure}
	\mfincludegraphics[width=\textwidth]{figures/l30lem125a.tikz}

	\caption{\label{l30:lem125a} The choice of contour when $y > 1$.}
\end{marginfigure}

\begin{proof}
	Let us consider $y > 1$ first.
	The integrand $\frac{y^w}{w}$ is analytic except for a simple pole at $w = 0$ with residue $1$, so integrating over the semicircle described in \autoref{l30:lem125a} by \nameref{thm4.2}
	\[
		1 = \frac{1}{2 \pi i} \int_{\gamma_R} y^w \frac{d w}{w} + \frac{1}{2 \pi i} \int_{c - i R}^{c + i R} y^w \frac{d w}{w}.
	\]
	Sending $R \to \infty$ the second integral becomes the integral we want, so ideally the first integral goes to $0$ as $R \to \infty$.

	So for $w \in \Set{\gamma_R}$ we parametrise as $w = c + R e^{i \theta}$, with $\frac{\pi}{2} \leq \theta \leq \frac{3 \pi}{2}$.
	Here also $d w = R i e^{i \theta} \, d \theta$, and we estimate $\abs{w} = \abs{c + R e^{i \theta}} \geq R - c$ (since $c$ is fixed and $R$ is large).
	Then
	\[
		\abs[\Big]{\frac{1}{2 \pi i} \int_{\gamma_R} y^w \frac{d w}{w}} \leq \frac{1}{2 \pi} \int_{\pi/2}^{3 \pi/2} y^c y^{R \cos(\theta)} \frac{R \, d \theta}{R - c}.
	\]
	Since $\frac{\pi}{2} \leq \theta \leq \frac{3 \pi}{2}$, $\cos(\theta) < 0$, and so $R \cos(\theta)$ is negative meaning that, since $y > 1$, $y^{R \cos(\theta)} \to 0$ as $R \to \infty$.
	Consequently the entire integral goes to $0$ since $\frac{R}{R - c} \to 1$ as $R \to \infty$.

	Next we consider the case $0 < y < 1$.
	The estimate is similar, only this time we want to integrate over a semicircle extending to the right (see \autoref{l30:lem125b}), since for $0 < y < 1$ we want positive powers to make the integrand small.

	\begin{marginfigure}
		\mfincludegraphics[width=\textwidth]{figures/l30lem125b.tikz}

		\caption{\label{l30:lem125b} The choice of contour when $0 < y < 1$.}
	\end{marginfigure}

	Again there is a simple pole of $\frac{y^w}{w}$ at $w = 0$, but this lies outside of our contour, so by \nameref{thm4.2}
	\[
		0 = \frac{1}{2 \pi i} \int_{\gamma_R} y^w \frac{d w}{w} + \frac{1}{2 \pi i} \int_{c - i R}^{c + i R} y^w \frac{d w}{w}.
	\]
	As before, the second integral goes to the integral we want when $R \to \infty$, and we show that the first integral goes to $0$.

	Indeed, with the same parametrisation, only with $-\frac{\pi}{2} \leq \theta \leq \frac{\pi}{2}$, we get
	\[
		\abs[\Big]{\frac{1}{2 \pi i} \int_{\gamma_R} y^w \frac{d w}{w}} \leq \frac{1}{2 \pi} \int_{\pi/2}^{-\pi/2} y^c y^{R \cos(\theta)} \frac{R \, d \theta}{R - c}.
	\]
	Since $\cos(\theta) > 0$ in this range, the exponent is positive, but $0 < y < 1$ so $y^{R \cos(\theta)} \to 0$ as $R \to \infty$.
	Hence the integral goes to $0$ as $R \to \infty$, again.
\end{proof}

In order to prove \nameref{thm12.7} we need the following technical lemma, the proof of which is not difficult, but is a bit lengthy:

\begin{lemma}\label{lem12.6}
	Let
	\[
		D(s) = \sum_{n = 1}^\infty \frac{a_n}{n^s}
	\]
	with abscissa of convergence $\sigma_c$.
	For $s = \sigma + i t$, $\sigma_c < \sigma < \sigma_c + 1$, we have
	\[
		D(s) \ll \abs{t}^{1 - (\sigma - \sigma_c) + \varepsilon}
	\]
	for any $\varepsilon > 0$.
\end{lemma}

The idea of the proof is similar to that of \autoref{prop12.2}.

\index{Perron's formula|(}
\begin{theorem}[Perron's formula]\label{thm12.7}
	Let
	\[
		D(s) = \sum_{n = 1}^\infty \frac{a_n}{n^s}
	\]
	with abscissa of convergence $\sigma_c$.
	Let $X > 0$, $X \not\in \N$, and $c > 0$.
	For any $s = \sigma + i t$ with $\sigma + c > \sigma_c$, we have
	\[
		\sum_{n < X} \frac{a_n}{n^s} = \frac{1}{2 \pi i} \int_{(c)} D(s + w) X^w \frac{d w}{w}.
	\]
	In particular,
	\[
		\sum_{n < X} a_n = \frac{1}{2 \pi i} \int_{(c)} D(w) X^w \frac{d w}{w}
	\]
	for $c > \sigma_c$.
\end{theorem}
\index{Perron's formula|)}

This is a remarkable formula: it translates a discrete average into an analytic formula.
Hence by understanding analytic properties of the Dirichlet series, we can glean understanding about the sequence $\Set{a_n}$.

\begin{proof}[Sketch of proof]
	The main idea of the proof is to write $D(s + w)$ in the right-hand side in terms of its series representation, then switch the order or summation and integration, like so:
	\begin{align*}
		\frac{1}{2 \pi i} \int_{(c)} D(s + w) X^w \frac{d w}{w} &= \frac{1}{2 \pi i} \int_{(c)} \Bigl( \sum_{n = 1}^\infty \frac{a_n}{n^{s + w}} \Bigr) X^w \frac{d w}{w} \\
		&= \sum_{n = 1}^\infty \frac{a_n}{n^s} \frac{1}{2 \pi i} \int_{(c)} \Bigl( \frac{X}{n} \Bigr)^w \frac{d w}{w}.
	\end{align*}
	There are two steps of note here: first, we can write $D(s + w)$ in terms of its series representation since $\Re(s + w) = \sigma + c > \sigma_c$, so we are in the region of convergence.
	Second, being able to switch the order of summation and integration is not trivial.
	In this case we can do this because we have control of the size of the tails courtesy of \autoref{lem12.6}.

	With those details out of the way, we are done: the resulting integral at the end can be evaluated using \autoref{lem12.5}:
	\[
		\frac{1}{2 \pi i} \int_{(c)} \Bigl( \frac{X}{n} \Bigr)^w \frac{d w}{w} = \begin{cases}
			1, \text{if $X > n$} \\
			0, \text{if $X < n$}.
		\end{cases}
	\]
	Hence this truncates the series to
	\[
		\frac{1}{2 \pi i} \int_{(c)} D(s + w) x^w \frac{d w}{w} = \sum_{n < X} \frac{a_n}{n^s}
	\]
	as desired.
\end{proof}

\begin{remark}
	Note that both \autoref{lem12.5} and \nameref{thm12.7} can be worked out in the case where $y = 1$ (or correspondingly $X \in \N$).
	In this case the integral in \autoref{lem12.5} evaluates to $\frac{1}{2}$ (from plain calculation, no tricky integration required), and consequently the resulting sum in \nameref{thm12.7} would add only half of the final term if $X \in \N$.

	All by way of saying: requiring $X \not\in \N$ simply makes the formula slightly cleaner looking; it isn't a technical limitation.
\end{remark}
