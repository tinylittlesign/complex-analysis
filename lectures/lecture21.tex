%!TEX root = ../lectures.tex

\lecture[October 29th, 2019]{Rank and Genus}

\topic{Jensen's formula}

Suppose $f(z)$ is an entire function.
By \nameref{thm8.8} we can write
\[
	f(z) = z^m \exp(g(z)) \prod_{n = 1}^\infty E_{p_n}\Bigl(\frac{z}{a_n}\Bigr).
\]
The choice of $p_n$ depends on how fast $\abs{a_n} \to \infty$, because we need
\[
	\sum_{n = 1}^\infty \Bigl( \frac{r}{\abs{a_n}} \Bigr)^{p_n + 1} < \infty.
\]
So if $\abs{a_n} \to \infty$ very fast, we can pick $p_n$ quite small.
That is to say, we want to control the growth rate of zeros of $b$, and we will show that the growth rate of zeros of $f$ in $B(0, r)$, as $r \to \infty$, is related to the growth rate of
\[
	M(r) = \sup_{0 \leq \theta \leq 2 \pi} \abs{f(r e^{i \theta})}.
\]

The setup for this is not too complicated: let $f$ be an analytic function in a neighbourhood of $\closure{B(0, r)}$, and suppose $f$ does not vanish on $\closure{B(0, r)}$.
Then we can define the logarithm of $f(z)$, and it is analytic on $B(0, r)$ (that is to say, there exists an analytic function $g(z)$ such that $f(z) = e^{g(z)}$).

The mean value property of analytic functions, this means
\[
	\log f(0) = \frac{1}{2 \pi} \int_0^{2 \pi} \log f(r e^{i \theta}) \, d \theta.
\]
Taking real parts, this becomes
\[
	\log \abs{f(0)} = \frac{1}{2 \pi} \int_0^{2 \pi} \log \abs{f(r e^{i \theta})} \, d \theta,
\]
if $f$ is non-vanishing.
The crucial insight here is that the left-hand side is fixed, and the integrand in the right-hand side is bounded by $M(r)$.

If $f$ does have zeros, one's first thought is to divide them away and study a new, non-vanishing function.
This is essentially the right idea, except dividing by $z - a_n$ directly does not give us any control of the size of those factors.
To remedy this, we instead divide by carefully chosen Möbius transformations, the growth of which we have very good control over.
This gives rise to:

\index{Jensen's formula|(}
\begin{theorem}[Jensen's formula]\label{thm8.11}
	Let $f(z)$ be an analytic function on a neighbourhood of $\closure{B(0, r)}$.
	Let $a_1, a_2, \dots, a_n$ be the zeros of $f$ in $B(0, r)$, repeated according to multiplicity.\footnote{There are only finitely many zeros on $B(0, r)$, since otherwise they we would have an infinite bounded sequence of zeros, so they would have a limit point, forcing $f$ to be identically zero.}
	Suppose $f(0) \neq 0$ and $f(z) \neq 0$ for all $\abs{z} = r$.
	Then
	\[
		\log \abs{f(0)} = - \sum_{k = 1}^n \log \Bigl(\frac{r}{\abs{a_n}} \Bigr) + \frac{1}{2 \pi} \int_0^{2 \pi} \log \abs{f(r e^{i \theta})} \, d \theta.
	\]
\end{theorem}
\index{Jensen's formula|)}

\begin{proof}
	Recall how for $\abs{\alpha} < 1$, the Möbius transformation
	\[
		\varphi_\alpha(z) = \frac{z - \alpha}{1 - \conjugate{\alpha} z}
	\]
	maps the unit disk $D$ to $D$, and the unit circle $\partial D$ to $\partial D$, and moreover $\varphi_\alpha(\alpha) = 0$.
	We want to use this on $B(0, r)$, so consider the diagram
	\[
		\begin{tikzcd}
			D \arrow[r, "\varphi_\alpha"] & D \\
			B(0, r). \arrow[u] \arrow[ur, dashed, "\Psi_a"']
		\end{tikzcd}
	\]
	The natural way to move from $B(0, r)$ to $D$ is to divide by $r$, so we define
	\begin{align*}
		\Psi_a(z) &\coloneqq \varphi_\alpha\Bigl(\frac{z}{r}\Bigr) = \frac{\frac{z}{r} - \alpha}{1 - \conjugate{\alpha} \frac{z}{r}} \\
		&= \frac{z - \alpha r}{r - \conjugate{\alpha} z} = \frac{r(z - a)}{r^2 - \conjugate{a} z},
	\end{align*}
	where we have relabelled $\alpha r = a$, so that $a \in B(0, r)$.
	Then $\Psi_a(z) \colon B(0, r) \to D$, and it maps $\partial B(0, r)$ to $\partial D$.
	This is useful, because it means the magnitude of $\Psi_a(z)$ for $\abs{z} = r$ is exactly $1$.

	Now define
	\[
		F(z) = f(z) \prod_{k = 1}^n \frac{r^2 - \conjugate{a_k} z}{r(z - a_k)}.
	\]
	This has removable singularities at $z = a_k$, so just let $F(z)$ represent the analytic function on $B(0, r)$ we get by removing those singularities.
	Moreover $F(z) \neq 0$ in $\closure{B(0, r)}$, by construction, and therefore
	\[
		\log \abs{F(0)} = \frac{1}{2 \pi} \int_0^{2 \pi} \log \abs{F(r e^{i \theta})} \, d \theta.
	\]
	Replacing $F(z)$ per above, this yields
	\[
		\log \abs[\Big]{f(0) \prod_{k = 1}^n - \Bigl( \frac{r}{a_k} \Bigr)} = \frac{1}{2 \pi} \int_0^{2 \pi} \log \abs[\Big]{f(r e^{i \theta}) \prod_{k = 1}^n \Psi_{a_k}(r e^{i \theta}) } \, d \theta.
	\]
	By the discussion above, $\abs{\Psi_{a_k}(r e^{i \theta})} = 1$; the product on the left-hand side becomes a sum when taken out of the logarithm, so
	\[
		\log \abs{f(0)} + \sum_{k = 1}^n \log \abs[\Big]{\frac{r}{a_k}} = \frac{1}{2 \pi} \int_0^{2 \pi} \log\abs{f(r e^{i \theta})} \, d \theta.
	\]
	Moving the sum to the other side completes the proof.
\end{proof}

\begin{exercise}
	In the hypothesis of \nameref{thm8.11}, do \emph{not} suppose that $f(0) \neq 0$.
	Show that if $f$ has a zero at $z = 0$ of multiplicity $m$, then
	\[
		\log\abs[\Big]{\frac{f^{(m)}(0)}{m!}} + m \log r = - \sum_{k = 1}^n \log \Bigl ( \frac{r}{\abs{a_k}} \Bigr) + \frac{1}{2 \pi} \int_0^{2 \pi} \log \abs{f(r e^{i \theta})} \, d \theta. \qedhere
	\]
\end{exercise}

\begin{exercise}\label{hw8.2}
	Let $f$ be an entire function with $f(0) \neq 0$, and let
	\[
		M(r) = \sup_{0 \leq \theta \leq 2 \pi} \abs{f(r e^{i \theta})}.
	\]
	Let $n(r)$ denote the number of zeros of $f$ in $B(0, r)$, counted according to multiplicities.
	Suppose $f(0) = 1$ (else normalise, since $f(0) \neq 0$).
	Then
	\[
		n(r) \log 2 \leq \log M(2 r). \qedhere
	\]
\end{exercise}

\topic{The genus and rank of entire functions}

Suppose $f$ is entire.
Again by \nameref{thm8.8}, we can write
\[
	f(z) = z^m \exp(g(z)) \prod_{n = 1}^\infty E_{p_n}\Bigl( \frac{z}{a_n} \Bigr),
\]
where $g(z)$ is an entire function.

Let us consider, in some sense, the simplest form this can take (without becoming trivial).
The first nontrivial entire functions are polynomials, so we are interested in the case where $g(z)$ is a polynomial.

Similarly, the simplest form $p_n$ can take is constant---so $p_n = p$ for all $n$.
This is equivalent to
\[
	\sum_{n = 1}^\infty \frac{1}{\abs{a_n}^{p + 1}} < \infty,
\]
since in the original condition of \nameref{thm8.8} we have
\[
	\sum_{n = 1}^\infty \Bigl( \frac{r}{\abs{a_n}} \Bigr)^{p_n + 1} < \infty,
\]
but with $p_n = p$ a constant, $r^{p + 1}$ is a constant, so we can take that out of the sum.

\begin{definition}[Rank]
	Let $f \in H(\C)$ with nonzero zeros $\Set{a_1, a_2, \dots}$, repeated according to multiplicities, and arranged such that $0 < \abs{a_1} \leq \abs{a_2} < \dots$.
	We say that $f$ is of \keyword{finite rank}\index{rank!finite} if there exists $p \in \N \cup \Set{0}$ such that
	\[
		\sum_{n = 1}^\infty \frac{1}{\abs{a_n}^{p + 1}} < \infty.
	\]
	If $p$ is the smallest nonnegative integer such that this condition holds, then we say that $f$ is of \keyword{rank $p$}\index{rank}.
\end{definition}

\begin{remark}
	\begin{items}
		\item If $f$ has only finitely many zeros, then the sum above always converges, and in particular converges for $p = 0$, so we say that $f$ has rank $0$.

		\item We say that $f$ is of \keyword{infinite rank}\index{rank!infinite} if it is not of finite rank.
		In other words, it is not of rank $p$ for any $p$.
	\end{items}
\end{remark}

\begin{remark}
	Suppose $f$ is of finite rank $p$.
	Then
	\[
		f(z) = z^m \exp(g(z)) \prod_{n = 1}^\infty E_p\Bigl( \frac{z}{a_n} \Bigr).
	\]
	Here $m$ is the order of vanishing of $f$ at $z = 0$, so $m$ is unique.
	Similarly, the $E_p$ terms are now uniquely determined, since we have chosen $p$ as small as possible.
	Finally, $\exp(g(z))$ is also unique, meaning that the entire expression is unique except $g(z)$ may be replaced by $g(z) + 2 \pi i k$ for $k \in \Z$.

	Finally, we call
	\[
		P(z) = \prod_{n = 1}^\infty E_p\Bigl( \frac{z}{a_n} \Bigr)
	\]
	the \keyword{standard form}\index{standard form} of $f$.
\end{remark}

\begin{definition}[Genus]
	An entire function $f$ has \keyword{finite genus}\index{genus!finite} if $f$ has finite rank and $g(z)$ is a polynomial, where $f(z) = z^m \exp(g(z)) P(z)$ and $P(z)$ is the standard form of $f$.

	Let $p$ be the rank of $f$ and $q$ be the degree of $g(z)$.
	Then $\mu \coloneqq \max\Set{p, q}$ is called the \keyword{genus}\index{genus} of $f$.
\end{definition}

Looking at this definition, this tells us that the genus of an entire function controls the factorisation.
As it turns out, the genus also controls the growth of the function:

\begin{theorem}\label{thm8.12}
	Let $f$ be an entire function of finite genus $\mu$.
	Then for any $\varepsilon > 0$,
	\[
		f(z) \ll_{\varepsilon, f} \exp(\abs{z}^{\mu + 1 + \varepsilon}).
	\]
	In particular, we have for some $M > 0$
	\[
		\log \abs{E_\mu(z)} \leq M \abs{z}^{\mu + 1}
	\]
	as well as
	\[
		\log \abs{E_\mu(z)} \leq M \abs{z}^{\mu}
	\]
	for all $z \in \C$.
\end{theorem}

\begin{remark}
	By the notation $f \ll g$ we mean that there exists some $c > 0$ such that $\abs{f(z)} \leq c \abs{g(z)}$ for all $z \in \C$, known as \keyword{Vinogradov notation}\index{Vinogradov notation}.

	Note that, in the case of smooth $f$ and $g$, this means we only need to consider $\abs{z}$ large, since on bounded sets $f$ and $g$ are both bounded, so we can always find a $c$ sufficiently large.
\end{remark}

\begin{remark}
	Which one of the bounds on $\log \abs{E_\mu(z)}$ from the theorem one wants to use depends on $\abs{z}$---if $\abs{z} < 1$ we would prefer a bigger power, since that results a smaller bound, and conversely for $\abs{z} \geq 1$ we would want a smaller power.
\end{remark}

% \begin{remark}
% 	It is instructive to note that when dealing with with logarithms and exponentials in this sense, it sometimes serves to be a little big careful.
% 	By definition,
% \end{remark}

\begin{proof}
	Since $f(z)$ is of finite genus $\mu$,
	\[
		f(z) = z^m \exp(g(z)) \prod_{n = 1}^\infty E_\mu\Bigl( \frac{z}{a_n} \Bigr),
	\]
	where $g(z)$ is a polynomial of degree at most $\mu$.
	Taking logarithms, for $z \not\in \Set{a_n} \cup \Set{0}$,
	\[
		\log \abs{f(z)} = m \log \abs{z} + \Re g(z) + \sum_{n = 1}^\infty \log \abs[\Big]{E_\mu \Bigl( \frac{z}{a_n} \Bigr)}.
	\]
	We proceed to bound these three parts one at a time.

	First, $\log \abs{z} \leq M_1 \abs{z}^{\varepsilon}$ for any $\varepsilon > 0$, so definitely $\log \abs{z} \leq M_1 \abs{z}^{\mu + 1}$ for some $M_1 > 0$ and $\abs{z}$ large.

	Second, $\abs{\Re g(z)} \leq \abs{g(z)} \leq M_2 \abs{z}^\mu$ since $g(z)$ is a polynomial of degree at most $\mu$, so in addition $\abs{\Re g(z)} \leq M_2 \abs{z}^{\mu + 1}$ for some $M_2 > 0$ and $\abs{z}$ large.

	Third, and most important, we study $\log \abs{E_\mu(z)}$ as set out in the statement of the theorem.
	First, let $\abs{z} < \frac{1}{2}$.
	Here since $\abs{z}$ is small, we have the power series expansion
	\[
		\log(1 - z) = - \Bigl( z + \frac{z^2}{2} + \dots + \frac{z^\mu}{\mu} + \dots \Bigr),
	\]
	and so since
	\[
		E_\mu(z) = (1 - z) \exp\Bigl( z + \frac{z^2}{2} + \dots + \frac{z^\mu}{\mu} \Bigr)
	\]
	we get
	\begin{align*}
		\log \abs{E_\mu(z)} &= \log\abs{1 - z} + \Re\Bigl( z + \frac{z^2}{2} + \dots + \frac{z^\mu}{\mu}\Bigr) \\
		&= -\Re\Bigl( \frac{z^{\mu + 1}}{\mu + 1} + \frac{z^{\mu + 2}}{\mu + 2} + \dots \Bigr) \\
		&= \abs{z}^{\mu + 1} ( 1 + \abs{z} + \abs{z}^2 + \dots )
	\end{align*}
	where we have bounded the real part by the modulus, and noted that the denominators in the middle step are all bigger than $1$.
	But now $\abs{z} < \frac{1}{2}$, so this is bounded by a geometric sum, namely
	\[
		\log \abs{E_\mu(z)} \leq \abs{z}^{\mu + 1} \Bigl( 1 + \frac{1}{2} + \frac{1}{2^2} + \dots \Bigr) = 2 \abs{z}^{\mu + 1}.
	\]
	Since $\abs{z} < \frac{1}{2} < 1$, making the power smaller enlarges the bound, so this also says $\log\abs{E_\mu(z)} \leq 2 \abs{z}^\mu$.

	On the other hand, for $\abs{z} > 2$, we have
	\[
		\log \abs{E_\mu(z)} \leq \log\abs{1 - z} + \abs{z} + \frac{\abs{z}^2}{2} + \dots + \frac{\abs{z}^\mu}{\mu} \leq C \abs{z}^\mu
	\]
	for some constant $C > 0$ since, $\abs{z} > 2 > 1$ is large, the last term dominates.
	This time, $\abs{z}$ being large, means enlarging the power enlarges the bound, so we also have $\log \abs{E_\mu(z)} \leq C \abs{z}^{\mu + 1}$.

	Finally, consider $\frac{1}{2} \leq \abs{z} \leq 2$.
	Here, notice how $\log\abs{E_\mu(z)}$ is a continuous function, except at $z = 1$, and there
	\[
		\lim_{z \to 1} \abs{E_\mu(z)} = -\infty,
	\]
	meaning that $\log \abs{E_\mu(z)}$ is bounded above, say $\log\abs{E_\mu(z)} \leq D$ for some $D > 0$.
	Hence since $\abs{z}$ lives in a bounded range, we have
	\[
		\log\abs{E_\mu(z)} \leq D_a \abs{z}^a
	\]
	for any power $a$, so in particular for the powers $\mu$ and $\mu + 1$.

	% Third, and most important, we study $\log \abs{E_\mu(z)} \leq M_3 \abs{z}^{\mu + 1}$ for some $M_3 > 0$, $\abs{z}$ large.
	% To see this we need to recall that
	% \[
	% 	E_n(z) = (1 - z) \exp\Bigl( z + \frac{z^2}{2} + \dots + \frac{z^n}{n} \Bigr).
	% \]
	% Therefore
	% \begin{align*}
	% 	\abs{E_\mu(z)} &= \abs[\Big]{(1 - z) \exp\Bigl( z + \frac{z^2}{2} + \dots + \frac{z^n}{n} \Bigr)} \\
	% 	&\leq (1 + \abs{z}) \exp\Bigl( \abs{z} + \frac{\abs{z}^2}{2} + \dots + \frac{\abs{z}^\mu}{\mu} \Bigr),
	% \end{align*}
	% since the real part of a complex number is bounded by its magnitude (the argument of the exponential becomes its real part when taking absolute values).
	% Hence
	% \[
	% 	\log \abs{E_\mu(z)} \leq \log(1 + \abs{z}) + \abs{z} + \frac{\abs{z}^2}{2} + \dots + \frac{\abs{z}^\mu}{\mu},
	% \]
	% where the last term is dominant, so
	% \[
	% 	\log \abs{E_\mu(z)} \leq M_3 \abs{z}^{\mu + 1}
	% \]
	% for some $M_3 > 0$ and $\abs{z}$ large.
	% Note that in principle our bound holds for $\abs{z}^{\mu}$, but as we see below, having the power $\mu + 1$ is helpful, because

	Now
	\[
		\sum_{n = 1}^\infty \log \abs[\Big]{E_\mu\Bigl( \frac{z}{a_n} \Bigr )} \leq M_3 \sum_{n = 1}^\infty \Bigl( \frac{z}{\abs{a_n}} \Bigr)^{\mu + 1} = M_3 \abs{z}^{\mu + 1} \sum_{n = 1}^\infty \frac{1}{\abs{a_n}^{\mu + 1}},
	\]
	where the final sum converges since $f$ is of genus $\mu$.
	Hence
	\[
		\sum_{n = 1}^\infty \log \abs[\Big]{E_\mu\Bigl( \frac{z}{a_n} \Bigr )} \leq M_4 \abs{z}^{\mu + 1}
	\]
	for some $M_4 > 0$ and $\abs{z}$ large.

	Putting the three pieces together, this means
	\[
		\log \abs{f(z)} \leq K \abs{z}^{\mu + 1}
	\]
	for some $K > 0$ and $\abs{z}$ large, so
	\[
		f(z) \ll \exp(\abs{z}^{\mu + 1 + \varepsilon}),
	\]
	where the $\varepsilon > 0$ comes from absorbing $K$ into the exponential.
\end{proof}
