%!TEX root = ../lectures.tex

\lecture[November 7th, 2019]{Bloch's and Landau's Constants}

\topic{Proof of Bloch's theorem}

\begin{proof}
	The broad strategy is this: we want a new function with derivative as large as possible at some point $a$, so that we can apply \autoref{lem9.3}, and moreover we want its derivative and the upper bound of the function to be of similar size, so that in \autoref{lem9.2} we get a constant.

	To this end, let
	\[
		K(r) = \max_{\abs{z} = r} \abs{f'(z)}
	\]
	and let $h(r) = (1 - r) K(r)$.
	Then $h \colon \interval{0}{1} \to \R$ is continuous, $h(0) = 1$ and $h(1) = 0$, by construction.
	Let $r_0 = \sup\Set{r \given h(r) = 1}$.

	Notice three things: first, since $h$ is continuous, $h(r_0) = 1$.
	Second, $0 \leq r_0 < 1$ since $h(1) = 0 \neq 1$, and third, for any $r > r_0$, $h(r) < 1$ because of the supremum.

	Now take $a \in D$ with $\abs{a} = r_0$ such that $K(r_0) = \abs{f'(a)}$.
	This must be possible since the supremum on $\abs{z} = r_0$, which is compact, must be attained.
	Then
	\[
		1 = h(r_0) = (1 - r_0) K(r_0) = (1 - r_0) \abs{f'(a)},
	\]
	or in other words
	\[
		\abs{f'(a)} = \frac{1}{1 - r_0}.
	\]
	Our goal is to apply \autoref{lem9.3} near $z = a$, so consider a ball $B(a, \rho_0)$ where $\rho_0 \coloneqq \frac{1}{2} (1 - r_0)$.
	Notice how this by construction means
	\begin{equation}\label{lec24:fparho0}
		\abs{f'(a)} = \frac{1}{2 \rho_0}.
	\end{equation}
	Then for $\abs{z - a} < \rho_0$, we have
	\[
		\abs{z} < \abs{a} + \frac{1}{2} (1 - r_0) = r_0 + \frac{1}{2} (1 - r_0) = \frac{1}{2} (1 + r_0).
	\]
	Notice how $r_0 < \frac{1}{2}(1 + r_0)$ since $r_0 < 1$, and how therefore $h(\frac{1}{2} (1 + r_0)) < 1$.
	Hence for $\abs{z - a} < \rho_0$,
	\[
		\abs{f'(z)} \leq K\Bigl( \frac{1}{2}(1 + r_0)\Bigr) = \frac{h(\frac{1}{2} (1 + r_0))}{1 - \frac{1}{2} (1 + r_0)},
	\]
	where the inequality in the first step is the \nameref{thm6.1}, and the second step is the definition of $K(r)$.
	Now since $h(\frac{1}{2}(1 + r_0)) < 1$, this is bounded by
	\[
		\abs{f'(z)} < \frac{1}{\frac{1}{2} (1 - r_0)} = \frac{1}{\rho_0}
	\]
	for all $\abs{z - a} < \rho_0$.

	To apply \autoref{lem9.3} we want $\abs{f'(z) - f'(a)} < \abs{f'(a)}$, so let us compute the former: for $\abs{z - a} < \rho_0$,
	\[
		\abs{f'(z) - f'(a)} \leq \abs{f'(z)} + \abs{f'(a)} < \frac{1}{\rho_0} + \frac{1}{2 \rho_0} = \frac{3}{2 \rho_0}.
	\]
	Unfortunately this is clearly not less than $\abs{f'(a)} = \frac{1}{2 \rho_0}$---the disk is too large---so we need to shrink $\rho_0$.

	Toward this, consider
	\[
		\Psi(z) = \frac{2 \rho_0}{3} (f'(\rho_0 z + a) - f'(a)),
	\]
	so that $\Psi \colon D \to \C$.
	Now $\Psi(0) = 0$ and
	\[
		\abs{\Psi(z)} \leq \frac{2 \rho_0}{3} \frac{3}{2 \rho_0} = 1
	\]
	by the above calculations, so we can apply \nameref{thm6.2}, which tells us that $\abs{\Psi(z)} \leq \abs{z}$ for all $\abs{z} \leq 1$.
	Consequently
	\[
		\abs{f'(\rho_0 z + a) - f'(a)} \leq \frac{3}{2 \rho_0} \abs{z},
	\]
	and making the change of variables $w = \rho_0 z + a \in B(a, \rho_0)$, so $z = \frac{w - a}{\rho_0}$, we get
	\[
		\abs{f'(w) - f'(a)} \leq \frac{3 \abs{w - a}}{2 \rho_0^2}
	\]
	for $w \in B(a, \rho_0)$.
	We want this to be less than $\abs{f'(a)} = \frac{1}{2 \rho_0}$, so
	\[
		\frac{3 \abs{w - a}}{2 \rho_0^2} < \frac{1}{2 \rho_0}
	\]
	means
	\[
		\abs{w - a} < \frac{1}{3} \rho_0.
	\]

	Therefore we take $z \in S = B(a, \frac{1}{3} \rho_0)$ and there we get
	\[
		\abs{f'(z) - f'(a)} < \frac{3}{2 \rho_0} \frac{1}{3} \rho_0 = \frac{1}{2 \rho_0} = \abs{f'(a)},
	\]
	so we can apply \autoref{lem9.3}, guaranteeing that $f$ is one-to-one on $S$.

	It remains to show that $f(S)$ contains a disk of radius $\frac{1}{72}$.
	To accomplish this we want to apply \autoref{lem9.2}, meaning that we need a function $g$ with $g(0) = 0$, and we need information about its derivative at $0$ and a bound for it on a ball.

	Let $g \colon B(0, \frac{1}{3} \rho_0) \to \C$ be defined by $g(z) = f(z + a) - f(a)$, so that $g(0) = 0$ and $\abs{g'(0)} = \abs{f'(a)} = \frac{1}{2 \rho_0}$.

	To get a bound on $\abs{g(z)}$ we need use the Fundamental theorem of calculus, since this way we can leverage our knowledge of the derivative.
	Consider the line segment $\gamma = \interval{a}{z + a} \subset S = B(a, \frac{1}{3} \rho_0) \subset B(a, \rho_0)$, for which
	\[
		\abs{g(z)} = \abs[\Big]{\int_\gamma g'(w) \, d w} \leq \int_\gamma \abs{g'(w)} \, d w.
	\]
	By definition $g'(w) = f'(w + a)$, and we know that $\abs{f'(z)} < \frac{1}{\rho_0}$ for all $\abs{z - a} < \rho_0$, and hence also $\abs{f'(w + a)} < \frac{1}{\rho_0}$, so that
	\[
		\abs{g(z)} \leq \frac{1}{\rho_0} \abs{z} < \frac{1}{\rho_0} \frac{1}{3} \rho_0 = \frac{1}{3}.
	\]
	Therefore by \autoref{lem9.2}, $g(B(0, \frac{1}{3} \rho_0)) \supset B(0, \sigma)$ with
	\[
		\sigma = \frac{(\frac{1}{3} \rho_0)^2 (\frac{1}{2 \rho_0})^2}{6 \cdot \frac{1}{3}} = \frac{\frac{1}{9} \cdot \frac{1}{4}}{2} = \frac{1}{72}.
	\]
	Shifting back to $f$, this means
	\[
		f(S) \supset B\Bigl( f(a), \frac{1}{72} \Bigr),
	\]
	finishing the proof.
\end{proof}

We can readily translate this to other disks centred at $z = 0$:

\begin{corollary}\label{cor9.5}
	Let $f$ be an analytic function on a neighbourhood of $\closure{B(0, R)}$.
	Then $f(B(0, R))$ contains a disk of radius $\frac{1}{72} R \abs{f'(0)}$.
\end{corollary}

\begin{proof}
	If $f'(0) = 0$, then the result is trivially true, so assume $f'(0) \neq 0$.
	Consider the function
	\[
		g(z) = \frac{f(R z) - f(0)}{R f'(0)},
	\]
	where $R z$ serves to move us from $B(0, R)$ to $D$, subtracting $f(0)$ is to make $g(0) = 0$, and dividing by $R f'(0)$ makes $g'(0) = 1$.

	Hence we can apply \nameref{thm9.4} on $g$, so that $g(D)$ contains a disk of radius $\frac{1}{72}$, and so $f(B(0, R))$ contains a disk of radius $\frac{1}{72} R \abs{f'(0)}$.
\end{proof}

The constant $\frac{1}{72}$ in \nameref{thm9.4} is not best possible.

\begin{definition}[Bloch's constant]
	Let $\mathcal{F}$ be the set of all functions $f$ that are analytic on a neighbourhood of $D = \Set{z \given \abs{z} < 1}$, with $f(0) = 0$ and $f'(0) = 1$.

	For each $f \in \mathcal{F}$, let $\beta(f)$ denote the supremum of all $r$ such that there exists a disk $S \subset D$ where $f$ is one-to-one on $S$ and $f(S)$ contains a disk of radius $r$.

	\keyword{Bloch's constant}\index{Bloch's constant} is the number $B$ defined by
	\[
		B = \inf_{f \in \mathcal{F}} \Set{\beta(f)}.
	\]
\end{definition}

\begin{remark}
	Notice how\nameref{thm9.4} implies that $B \geq \frac{1}{72}$.
	On the other hand, considering the function $f(z) = z$, we see that $\beta(f) = 1$, so $B \leq 1$.

	The state of the art is
	\[
		0.4332\approx\frac{\sqrt{3}}{4}+3\times10^{-4}\leq B\leq \sqrt{\frac{\sqrt{3}-1}{2}} \cdot \frac{\Gamma(\frac{1}{3})\Gamma(\frac{11}{12})}{\Gamma(\frac{1}{4})}\approx 0.4719,
	\]
	the lower bound due to Chen and Gauthier in \cite{Chen1996} and then marginally improved by Xiong in \cite{Xiong1998}, and the upper bound is due to Ahlfors and Grunsky in \cite{Ahlfors1937}.

	It is further conjectured in the Ahlfors and Grunsky paper is the true value of $B$.
\end{remark}

We can ask a related, but slightly relaxed question:

\begin{definition}[Landau's constant]
	For each $f \in \mathcal{F}$, define $\lambda(f)$ to be the supremum of all $r$ such that $f(D)$ contains a disk of radius $r$.
	\keyword{Landau's constant}\index{Landau's constant} $L$ is defined by
	\[
		L = \inf_{f \in \mathcal{F}} \Set{\lambda(f)}.
	\]
\end{definition}

\begin{remark}
	Naturally $\lambda(f) \geq \beta(f)$ since $\lambda$ drops the requirement of $f$ being one-to-one on the disk in question.
	Hence since $\lambda(f) \geq \beta(f)$ for all $f \in \mathcal{F}$ we must have $B \geq L$, and as before, taking $f(z) = z$, $L \geq 1$.

	The best known bounds for $L$ are
	\[
		0.5 < L \leq \frac{\Gamma(\frac{1}{3}) \Gamma(\frac{5}{6})}{\Gamma(\frac{1}{6})} \approx 0.5433
	\]
	Note that in particular this tells us that $B < L$, so the equality is ruled out.
\end{remark}

Notice how, being defined by an infimum, it is possible, in principle, no function ever attains $L$ exactly.
As it happens, this is not actually the case:

\begin{proposition}\label{prop9.6}
	Let $f$ be analytic on a neighbourhood of $\closure{D}$ with $f(0) = 0$ and $f'(0) = 1$.
	Then $f(D)$ contains a disk of radius $L$.
\end{proposition}

\begin{proof}
	We will show something slightly stronger, namely that $f(D)$ contains a disk of radius $\lambda = \lambda(f)$, and since $\lambda(f) \geq L$, this implies the proposition.

	By the definition of $\lambda(f)$, in terms of a supremum, we have that for each $n \in \N$ there exists some $\alpha_n \in f(D)$ such that $f(D) \supset B(\alpha_n, \lambda - \frac{1}{n})$.

	Notice now how $\Set{\alpha_n} \subset f(D) \subset f(\closure{D})$.
	Crucially, $\closure{D}$ is compact, and $f$ is continuous, whence maps compact sets to compact sets, so $f(\closure{D})$ is compact.
	Therefore $\Set{\alpha_n}$ has a limit point.

	For instance, take a subsequence $\Set{\alpha_{n_k}}$ such that $\alpha_{n_k} \to \alpha \in f(\closure{D})$.
	Then we claim that $f(D) \supset B(\alpha, \lambda)$.

	This is an exercise in the triangle inequality: for $w \in B(0, \lambda)$, choose $M$ large enough so that $\abs{w - a} < \lambda - \frac{1}{M}$.

	\begin{marginfigure}
		\mfincludegraphics[width=\textwidth]{figures/l24fig96.tikz}

		\caption{\label{prop96:fig} The setup of $\alpha$, $w$, and $\alpha_{n_k}$.}
	\end{marginfigure}

	Since $\alpha_{n_k} \to \alpha$, there must exist some $N \in \N$ large enough so that $N > 2 M$ and for $n_k > N$, $\abs{\alpha_{n_k} - \alpha} < \frac{1}{2 M}$.
	Then
	\[
		\abs{w - \alpha_{n_k}} \leq \abs{w - \alpha} + \abs{\alpha -\alpha_{n_k}} < \lambda - \frac{1}{M} + \frac{1}{2 M} = \lambda - \frac{1}{2 M} < \lambda - \frac{1}{n_k}
	\]
	since $2 M < N < n_k$.
	Hence $w \in B(\alpha_{n_k}, \lambda - \frac{1}{n_k}) \subset f(D)$, so $B(\alpha, \lambda) \subset f(D)$.
\end{proof}

Again we can translate this to arbitrary disks centred on $z = 0$ (by exactly the same method \autoref{cor9.5}):

\begin{corollary}\label{cor9.7}
	Let $f$ be analytic on a neighbourhood of $\closure{B(0, R)}$.
	Then $f(B(0, R))$ contains a disk of radius $\abs{f'(0)} R L$.
\end{corollary}

\begin{proof}
	As in \autoref{cor9.5}, $f'(0) = 0$ is trivial, so assume $f'(0) \neq 0$ and consider
	\[
		g(z) = \frac{f(R z) -f(0)}{R f'(0)},
	\]
	applying the \autoref{prop9.6}.
\end{proof}

\topic{The Little Picard theorem}

We showed, as a consequence of \nameref{thm8.15}, a \nameref{thm8.16}, saying that a non-constant entire function of finite order can miss at most one point in $\C$.
Our next goal is to show that the assumption of finite order is not necessary.

To accomplish this we need the following small calculation:

\begin{lemma}\label{lem9.8}
	Let $G$ be a simply connected region.
	Let $f \colon G \to \C$ be analytic.
	Suppose $f$ does not assume the values $0$ or $1$.
	Then there exists an analytic function $g \colon G \to \C$ such that
	\[
		f(z) = - \exp\left(\pi i \cosh(2 g(z))\right)
	\]
	for all $z \in G$.
\end{lemma}

It is worth recalling here that
\[
	\cosh(z) \coloneqq \frac{e^z + e^{-z}}{2}.
\]

\begin{proof}
	The proof essentially boils down to solving the expression for $f(z)$ above for $g(z)$, then working backwards.

	Since $f$ does not vanish on $G$, which is simply connected, we can take logarithms.
	By way of saying, there exists some analytic $h \colon G \to \C$ such that $f(z) = \exp(h(z))$.

	Let $F(z) = \frac{1}{2 \pi i} h(z)$, and notice how for any $n \in \Z$, we have $F(z) \neq n$ for all $z \in G$.
	If not, $h(z) = 2 \pi i n$, implying that $f(z) = \exp(h(z)) = 1$, which is a contradiction.

	In particular, $F(z)$ does not assume $0$ or $1$, so both $F(z)$ and $F(z) - 1$ have analytic square roots, so we can define
	\[
		H(z) = \sqrt{F(z)} - \sqrt{F(z) - 1},
	\]
	analytic on $G$.
	Moreover $H(z) \neq 0$ for all $z \in G$, since otherwise we would have $0 = \sqrt{F(z)} - \sqrt{F(z) - 1}$, which rearranged and squared gives us $0 = 1$.

	So there exists, finally, an analytic $g \colon G \to \C$ such that $H(z) = \exp(g(z))$, and therefore, retracing our steps,
	\begin{align*}
		\cosh(2 g(z)) + 1 &= \frac{e^{2 g(z)} + e^{-2 g(z)}}{2} + 1 = \frac{(e^{g(z)} + e^{-g(z)})^2}{2} \\
		&= \frac{(H(z) + \frac{1}{H(z)})^2}{2} = 2 F(z) = \frac{1}{\pi i} h(z).
	\end{align*}
	Hence
	\[
		f(z) = \exp(h(z)) = \exp(\pi i (\cosh(2 g(z)) + 1)) = - \exp\left(\pi i \cosh(2 g(z))\right)
	\]
	by pulling out the factor of $\exp(\pi i) = -1$.
\end{proof}
