%!TEX root = ../lectures.tex

\lecture[October 22nd, 2019]{Infinite Products}

\topic{Review of infinite products}

As hinted at toward the end of last lecture, we will soon have to deal with the matter of convergence of infinite products.
To this end we need to settle what that means.

\begin{definition}[Infinite product]
	Let $\Set{z_n} \subset \C$.
	If
	\[
		z = \lim_{k \to \infty} \prod_{n = 1}^k z_n
	\]
	exists, then we denote
	\[
		z = \prod_{n = 1}^\infty z_n
	\]
	as the \keyword{infinite product}\index{infinite product} of $z_n$.
\end{definition}

\begin{remark}
	Of course there are cases in which this is trivially zero.
	First, if $z_n = 0$ for some $n$, then $z = 0$.

	Similarly, if $z_n = a$ for all $n$ is constant, then if $\abs{a} < 1$,
	\[
		\prod_{n = 1}^\infty z_n = 0.
	\]
\end{remark}

Suppose, in view of this, $\Set{z_n} \subset \C \setminus \Set{0}$ and
\[
	\prod_{n = 1}^\infty z_n = z \neq 0.
\]
Then setting
\[
	P_k = \prod_{n = 1}^k z_n,
\]
we get $z_k = \frac{P_k}{P_{k - 1}}$.
Thus $z_k \to \frac{z}{z} = 1$ as $k \to \infty$, so for $n$ large enough, $z_n \neq 0$ (being close to $1$), so we can define its logarithm
\[
	\log z_n = \log\abs{z_n} + i \arg z_n,
\]
taking the branch $-\pi < \arg z_n < \pi$.

\begin{proposition}\label{prop8.1}
	Let $\Set{z_n} \subset \C$ such that $\Re(z_n) > 0$ for all $n$.\footnote{Note how this is a minor assumption: just consider the tail, in general, and it must be true if the product converges.}
	Then
	\[
		z = \prod_{n = 1}^\infty z_n \neq 0
	\]
	converges if and only if
	\[
		\sum_{n = 1}^\infty \log z_n,
	\]
	taking the branch $-\pi < \arg z_n < \pi$, converges.
\end{proposition}

\begin{proof}
	The meat of the proposition is the forwards direction.
	Write $z = r e^{i \theta}$, taking $-\pi < \theta \leq \pi$.
	Let
	\[
		P_k = \prod_{n = 1}^k z_n.
	\]
	Since $P_k \to z$ as $k \to \infty$, for $k$ large enough $\log_\theta P_k = \log\abs{P_k} + i \theta_k$, where by $\log_\theta$ we mean the branch cut is $\theta - \pi < \theta_k < \theta + \pi$, i.e., the branch opposite $z$.

	Let
	\[
		S_k = \sum_{n = 1}^k \log z_n
	\]
	with the branch cut $-\pi < \arg z_n < \pi$.
	Then
	\[
		\exp(S_k) = \prod_{n = 1}^k z_n = P_k.
	\]
	Taking logarithms, $S_k = \log_\theta P_k + 2 \pi i m_k$.
	Since $P_k \to z$ as $k \to \infty$,
	\begin{align*}
		S_k - S_{k - 1} &= \log_\theta P_k - \log_\theta P_{k - 1} + 2 \pi i (m_k - m_{k - 1}) \\
		&= \log \abs{z_k} + i (\theta_k - \theta_{k - 1}) + 2\pi i (m_k - m_{k - 1}).
	\end{align*}
	On the one hand, $S_k - S_{k - 1} = \log z_k \to \log 1 = 0$ as $k \to \infty$ since $z \to 1$.

	On the other hand, $\log \abs{z_k} \to \log 1 = 0$, and $\theta_k - \theta_{k - 1} \to 0$ since $P_k \to z$.
	Hence $m_k - m_{k - 1} \to 0$ as $k \to \infty$, but both $m_k$ and $m_{k - 1}$ are integers, so $m_k \to m$ for some integer $m$.

	Thus, as $k \to \infty$, $S_k \to \log_\theta z + 2 \pi i m$, i.e.,
	\[
		\sum_{n = 1}^\infty \log z_n
	\]
	converges.

	Note how all of this is just to make sure the angle $m_k$ converges, making sure it doesn't vary by multiples of $2 \pi i$ as $k$ changes.

	The reverse direction is trivial: let
	\[
		S_k = \sum_{n = 1}^\infty \log z_n.
	\]
	Assume $S_k \to s$ as $k \to \infty$.
	Then
	\[
		\exp(S_k) = \prod_{n = 1}^\infty z_n \to \exp(s) \neq 0. \qedhere
	\]
\end{proof}

\begin{remark}
	Since we need the factors to eventually be close to $1$, it is sometimes more convenient to consider
	\[
		\prod_{n = 1}^\infty (1 + z_n).
	\]
	The proposition then says that for $\Re(z_n) > -1$,
	\[
		\prod_{n = 1}^\infty (1 + z_n) \neq 0
	\]
	if and only if
	\[
		\sum_{n = 1}^\infty \log(1 + z_n)
	\]
	converges.
\end{remark}

Notice two things.
First, this means $z_n \to 0$ as $n \to \infty$, and second, $\log(1 + z) \approx z$ for $z$ near zero.
Hence:
\begin{exercise}\label{ex:lec19a}
	Let $\Re(z_n) > -1$.
	Then
	\[
		\sum_{n = 1}^\infty \log(1 + z_n)
	\]
	converges absolutely if and only if
	\[
		\sum_{n = 1}^\infty z_n
	\]
	converges absolutely.
\end{exercise}

This raises the question of how to correctly define absolute convergence for infinite products.

The first naïve attempt is to say that
\[
	\prod_{n = 1}^\infty z_n
\]
converges absolutely if
\[
	\prod_{n = 1}^\infty \abs{z_n} < \infty.
\]
However this is not a terribly useful definition, because the latter product converging does not imply the former does.
For instance, let $z_n = (-1)^n$.
Then
\[
	\prod_{n = 1}^\infty \abs{(-1)^n} = 1,
\]
but
\[
	\prod_{n = 1}^\infty (-1)^n
\]
does not converge---it oscillates between $-1$ and $1$.

A better choice, inspired by the above proposition, is:

\begin{definition}[Absolute convergence]
	Let $\Set{z_n} \subset \C$ with $\Re(z_n) > 0$ for all $n$.
	We say
	\[
		\prod_{n = 1}^\infty z_n
	\]
	\keyword{converges absolutely}\index{absolute convergence} if
	\[
		\sum_{n = 1}^\infty \log z_n
	\]
	converges absolutely.
\end{definition}

With this definition, fortunately, absolute convergence does imply ordinary convergence.

Combining this with the observations from \autoref{ex:lec19a}, we get

\begin{corollary}\label{cor8.2}
	Let $\Set{z_n} \subset \C$ with $\Re(z_n) > 0$ for all $n$.
	Then
	\[
		\prod_{n = 1}^\infty z_n
	\]
	converges absolutely if and only if
	\[
		\sum_{n = 1}^\infty (1 - z_n)
	\]
	converges absolutely.
\end{corollary}

Since our method for translating information from infinite series to infinite products is the exponential function, we naturally ask the following question:
Let $X$ be a metric space, and let $f_n \colon X \to \C$, $n = 1, 2, \dots$.
Suppose $f_n \to f$ uniformly on $X$.
Do we have
\[
	\exp(f_n) \to \exp(f)
\]
uniformly on $X$?
The answer, in general, is no---as it happens, though $f_n$ and $f$ are `close', the exponential can amplify these small differences a lot.

\begin{counterexample}
	Let $f_n(x) = \frac{x}{n} + \frac{1}{x}$ for $x \in \interval[open]{0}{1}$.
	As $n \to \infty$, $f_n(x) \to \frac{1}{x}$ uniformly on $\interval[open]{0}{1}$.

	However $\exp(f_n)$ does not converge uniformly to $\exp(f)$:
	\[
		\exp(f_n(x)) - \exp(f(x)) = \exp\Bigl ( \frac{1}{x} \Bigr ) \left ( \exp\Bigl( \frac{x}{n} \Bigr) - 1 \right ).
	\]
	Take $x = \frac{1}{n}$.
	Then
	\[
		\exp(f_n(x)) - \exp(f(x)) = \exp(n) \left ( \exp\Bigl( \frac{1}{n^2} \Bigr) - 1 \right ) \approx \exp(n) \frac{1}{n^2} \to \infty
	\]
	as $n \to \infty$.
	(To see this, write $\exp(\frac{1}{n^2})$ as its power series.)
	Hence this does not converge uniformly (though it does converge pointwise).
\end{counterexample}

In looking for a sufficient condition to alleviate this issue, consider $\exp(f_n) \to \exp(f)$ pointwise.
This means, roughly,
\[
	\abs[\Big]{\frac{\exp(f_n}{\exp(f)} - 1} < \varepsilon,
\]
and so multiplying by $\exp(f)$,
\[
	\abs{\exp(f_n) - \exp(f)} < \varepsilon \exp(f).
\]
Hence if $\exp(f)$ is uniformly bounded, then $\exp(f_n) \to \exp(f)$ does converge uniformly.
Formally:

\begin{lemma}\label{lem8.3}
	Let $X$ be a metric space, and let $f, f_n \colon X \to \C$, $n = 1, 2, \dots$.
	Suppose $f_n \to f$ uniformly on $X$.
	Suppose there exists some $a \in \R$ such that $\Re(f(x)) < a$ for all $x \in X$ (so $\abs{\exp(f(x))} = \exp(\Re(f(x)))$ is uniformly bounded).
	Then $\exp(f_n) \to \exp(f)$ uniformly on $X$.
\end{lemma}

\begin{proof}
	Since $e^z \to 1$ as $z \to 0$, for any $\varepsilon > 0$ there exists $\delta > 0$ such that if $\abs{z} < \delta$, then $\abs{e^z - 1} < \varepsilon e^{-a}$.

	Since $f_n \to f$ uniformly on $X$, there exists $N \in \N$ such that for $n > N$, $\abs{f_n(x) - f(x)} < \delta$ for all $x \in X$.
	Hence for $n > N$, $\abs{\exp(f_n(x) - f(x)) - 1} < \varepsilon e^{-a}$, so
	\[
		\abs[\Big]{\frac{\exp(f_n(x))}{\exp(f(x))} - 1} < \varepsilon e^{-a}.
	\]
	Multiplying both sides by $\exp(f(x))$, this becomes
	\[
		\abs{\exp(f_n(x)) - \exp(f(x))} < \varepsilon e^{-a} \abs{\exp(f(x))} \leq \varepsilon e^{-a} e^a = \varepsilon
	\]
	for all $x \in X$ since $\abs{\exp(f(x))} = \exp(\Re(f(x))) \leq e^a$ for all $x \in X$.
\end{proof}

\begin{lemma}\label{lem8.4}
	Let $X$ be a compact metric space.
	Let $g_n \colon X \to \C$, $n = 1, 2, \dots$, be continuous.
	Suppose
	\[
		\sum_{n = 1}^\infty g_n(x)
	\]
	converges absolutely and uniformly on $X$.
	Then
	\[
		f(x) \coloneqq \prod_{n = 1}^\infty (1 + g_n(x))
	\]
	converges absolutely and uniformly on $X$.

	Moreover, there exists an $N \in \N$ such that $f(x) = 0$ if and only if $f_n(x) = -1$ for some $n$ with $1 \leq n \leq N$.
\end{lemma}

\begin{proof}
	The series
	\[
		\sum_{n = 1}^\infty g_n(x)
	\]
	converging uniformly means there exists some $N \in \N$ such that for $n > N$, $\abs{g_n(x)} < \frac{1}{2}$ for all $x \in X$.
	Hence $\Re(1 + g_n(x)) > 0$ for all $n > N$, $x \in X$.
	Thus, for $n > N$, $\log(1 + g_n(x))$ is defined.

	Looking at the power series of $\log(1 + z)$, we see that $\abs{\log(1 + z)} \leq \frac{3}{2} \abs{z}$ for all $\abs{z} \leq \frac{1}{2}$, so moreover $\abs{\log(1 + g_n(x))} \leq \frac{3}{2} \abs{g_n(x)}$ for all $n > N$ and $x \in X$.

	Hence
	\[
		\sum_{n = N + 1}^\infty \log(1 + g_n(x)),
	\]
	being bounded by
	\[
		\frac{3}{2} \sum_{n = N + 1}^\infty g_n(x),
	\]
	converges absolutely and uniformly on $X$.
	Therefore
	\[
		h(x) \coloneqq \sum_{n = N + 1}^\infty \log(1 + g_n(x)),
	\]
	being the limit of a uniformly convergent sequence of continuous functions, is continuous.
	Since $X$ is compact, $h(x)$ is bounded on $X$, so $\Re(h(x)) < a$ for some $a \in \R$ for all $x \in X$.
	By \autoref{lem8.3}, taking exponentials,
	\[
		\exp(h(x)) = \prod_{n = N + 1}^\infty (1 + g_n(x))
	\]
	converges uniformly on $X$.

	Thus
	\[
		f(x) = (1 + g_1(x)) (1 + g_2(x)) \dots (1 + g_n(x)) \exp(h(x))
	\]
	converges absolutely and uniformly on $X$ since a finite number of factors doesn't affect convergence.

	Moreover, $\exp(h(x)) \neq 0$ for all $x \in X$ because of the exponential, hence if $f(x) = 0$, then $f_n(x) = -1$ for some $1 \leq n \leq N$ since the zero must come from the finite part.
\end{proof}

Translating this back to analytic functions on $\C$, this becomes
\begin{theorem}\label{thm8.5}
	Let $G \subset \C$ be a region.
	Let $\Set{f_n} \subset H(G)$ such that no $f_n$ is identically zero.
	Suppose
	\[
		\sum_{n = 1}^\infty (f_n(z) - 1)
	\]
	converges absolutely and uniformly on compact subsets of $G$.
	Then
	\[
		f(z) = \prod_{n = 1}^\infty f_n(z)
	\]
	converges and $f(z) \in H(G)$.

	Moreover, if $z = a$ is a zero of $f$, then $z = a$ is a zero of a finite number of $f_n(z)$, and the multiplicity of the zero of $f$ is the sum of the multiplicities of the zeros of $f_n$.
\end{theorem}
