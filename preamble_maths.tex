%!TEX root = lectures.tex

% Redefine equation environment to use gather, which is supported by amsmath (and therefore causes fewer spacing issues, and interacts better with hyperref)
\let\equation\gather
\let\endequation\endgather

% Define some common variously styled letters we occasionally need.
\newcommand{\C}{\ensuremath{\mathbb{C}}}
\newcommand{\Q}{\ensuremath{\mathbb{Q}}}
\newcommand{\R}{\ensuremath{\mathbb{R}}}
\newcommand{\Z}{\ensuremath{\mathbb{Z}}}
\newcommand{\N}{\ensuremath{\mathbb{N}}}
\newcommand{\F}{\ensuremath{\mathbb{F}}}
\newcommand{\OO}{\ensuremath{\mathcal{O}}}
\newcommand{\PP}{\ensuremath{\mathcal{P}}}
\newcommand{\M}{\ensuremath{\mathcal{M}}}
\newcommand{\Sw}{\ensuremath{\mathcal{S}}}
\newcommand{\HH}{\ensuremath{\mathbb{H}}}

\newcommand{\cont}{\ensuremath{\mathcal{C}}}

% a \pmod*{} variant with less spacing for subscripting.
\makeatletter
\let\@@pmod\pmod
\DeclareRobustCommand{\pmod}{\@ifstar\@pmods\@@pmod}
\def\@pmods#1{\mkern4mu({\operator@font mod}\mkern 6mu#1)}
\makeatother

% Allow overwriting of \Re and \Im.
\let\Re\relax
\let\Im\relax
% Redefine \Re and \Im as Re and Im instead of mathfrak letters.
\DeclareMathOperator{\Re}{Re}
\DeclareMathOperator{\Im}{Im}

\newcommand{\pmid}{\parallel}

\newcommand{\legendre}[2]{\genfrac{(}{)}{}{}{#1}{#2}}
% Spectra
\DeclareMathOperator{\Spec}{Spec}
\DeclareMathOperator{\mspec}{m-spec}

\DeclareMathOperator{\nilrad}{nilrad}
\DeclareMathOperator{\rad}{rad}

\DeclareMathOperator{\e}{e}
\DeclareMathOperator*{\Res}{Res}
\DeclareMathOperator{\Ann}{Ann}
\DeclareMathOperator{\Hom}{Hom}
\DeclareMathOperator{\End}{End}
\DeclareMathOperator{\Id}{Id}


\DeclareMathOperator{\interior}{int}

\DeclareMathOperator{\ord}{ord}
\DeclareMathOperator{\diam}{diam}

\DeclareMathOperator{\Aut}{Aut}

\DeclareMathOperator{\Li}{Li}

\DeclareMathOperator{\Vol}{Vol}

\DeclareMathOperator{\Gal}{Gal}
\DeclareMathOperator{\Cl}{Cl}

\DeclareMathOperator{\Tr}{Tr}
\DeclareMathOperator{\tr}{tr}
\DeclareMathOperator{\Norm}{N}
\DeclareMathOperator{\chr}{char}
\DeclareMathOperator{\disc}{disc}

\DeclareMathOperator{\SL}{SL}
\DeclareMathOperator{\PSL}{PSL}

\DeclareMathOperator{\coker}{coker}
\DeclareMathOperator{\supp}{supp}

% Declare an image symbol
\DeclareMathOperator{\im}{im}


\DeclareMathOperator{\lcm}{lcm}

% Define a placeholder for functions without arguments.
\newcommand*{\placeholder}{\makebox[1ex]{\textbf{$\cdot$}}}


% Define a plethora of theorem/definition/lemma/etc. environments.
% Number them all using the same counter.
\newtheorem{theorem}{Theorem}[subsection]
\newtheorem*{theorem*}{Theorem}
\newtheorem*{fact}{Fact}
\newtheorem*{claim}{Claim}
\newtheorem{corollary}[theorem]{Corollary}
\newtheorem{lemma}[theorem]{Lemma}
\newtheorem{proposition}[theorem]{Proposition}
\newtheorem{conjecture}[theorem]{Conjecture}

% For manually numbering a theorem, e.g. for use in restating a theorem.
% see https://tex.stackexchange.com/questions/391443/new-theorem-environment-with-manual-theorem-number
\newtheorem{manualtheoreminner}{Theorem}
\newenvironment{manualtheorem}[1]{%
  \IfBlankTF{#1}
    {\renewcommand{\themanualtheoreminner}{\unskip}}
    {\renewcommand\themanualtheoreminner{#1}}%
  \manualtheoreminner
}{\endmanualtheoreminner}

\theoremstyle{definition}
\newtheorem{definition}[theorem]{Definition}
\newtheorem{definitions}[theorem]{Definitions}
\newtheorem{notation}[theorem]{Notation}
\newtheorem{axiom}[theorem]{Axiom}
\newtheorem{problem}[theorem]{Problem}
\theoremstyle{remark}
\newtheorem{remark}[theorem]{Remark}


% Define special example(s) and solution(s) environments in order to have end-of-environment marks.
\declaretheorem[
	style = definition,
	qed = $\blacktriangle$,
	sibling = theorem
]{example}
\declaretheorem[
	style = definition,
	qed = $\blacktriangle$,
	sibling = theorem
]{examples}

\declaretheorem[
	style = definition,
	qed = $\blacktriangle$,
	sibling = theorem
]{counterexample}

\declaretheorem[
	style = remark,
	qed = $\blacklozenge$,
	numbered = no
]{solution}
\declaretheorem[
	style = remark,
	qed = $\blacklozenge$,
	numbered = no
]{solutions}

\declaretheorem[
	style = remark,
	qed = $\blacksquare$,
	numberwithin = section,
]{exercise}

\newcommand*{\vv}[1]{\bm{#1}}


% Redefine \overline to the nicely semantic \conjugate.
\newcommand*{\conjugate}[1]{\overline{#1}}

\newcommand*{\closure}[1]{\overline{#1}}

\newcommand*{\reduction}[1]{\overline{#1}}

\newcommand*{\completion}[1]{\widehat{#1}}

% For all of the below, use \command* to make the delimiters adjust size automatically.

% Defines an absolute value notation. Give it no argument and it'll default to \abs{\placeholder}.
\DeclarePairedDelimiterX{\abs}[1]{\lvert}{\rvert}{% \abs{a}
	\ifblank{#1}{\placeholder}{#1}%
}

\DeclarePairedDelimiterX{\floor}[1]{\lfloor}{\rfloor}{% \floor{a}
	\ifblank{#1}{\placeholder}{#1}%
}

\DeclarePairedDelimiterX{\fracpart}[1]{\{}{\}}{% \fracpart{a}
	\ifblank{#1}{\placeholder}{#1}%
}

% Defines a semantic notation for principal ideals.
\DeclarePairedDelimiterX{\principal}[1]{\langle}{\rangle}{% \principal{a}
	\ifblank{#1}{\placeholder}{#1}%
}

% Defines a norm notation. Give it no argument and it'll default to \norm{\placeholder}.
\DeclarePairedDelimiterX{\norm}[1]{\lVert}{\rVert}{% \norm{a}
	\ifblank{#1}{\placeholder}{#1}%
}

% Defines a scalar product notation. Takes two arguments,
% and replaces any missing argument with \placeholder.
\DeclarePairedDelimiterX{\scalar}[2]{\langle}{\rangle}{% \scalar{a}{b}
	\ifblank{#1}{%
		\ifblank{#2}{\placeholder, \placeholder}{#1, \placeholder}%
	}{#1, #2}%
}

% Provide semantic notation for describing sets with conditions.
\providecommand\given{\:\vert\:} % Just make sure this exists, so that things don't explode.
\DeclarePairedDelimiterX{\set}[1]{\{}{\}}{% \Set{ x \given x > 0}
	\,\renewcommand\given{\nonscript\:\delimsize\vert\nonscript\:\mathopen{}}#1\,
}

% The following makes \big the default for the \Set command:
\let\oldset\set
\def\set{\futurelet\testchar\MaybeOptArgSet}
\def\MaybeOptArgSet{\ifx[\testchar \let\next\OptArgSet
\else \let\next\NoOptArgSet \fi \next}
\def\OptArgSet[#1]#2{\oldset[#1]{#2}}
\def\NoOptArgSet#1{\OptArgSet[\big]{#1}}

\makeatletter
\newcommand{\Set}{%
             \@ifstar
                  {\oldset*}%
                  {\set}%
}
\makeatother

% Set the interval package to use parens for open intervals.
\intervalconfig{
	soft open fences
}

% Redefine \left and \right to use \mathopen{} and \mathclose{} appropriately, so as to correct spacing on, for instance, $\sin\left ( x \right )$.
\let\originalleft\left
\let\originalright\right
\renewcommand{\left}{\mathopen{}\mathclose\bgroup\originalleft}
\renewcommand{\right}{\aftergroup\egroup\originalright}
